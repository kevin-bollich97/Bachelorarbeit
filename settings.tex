%%%%%%%%%%%%%%%%%%%%%%%%%%%%%%%%%%%%
%           Packages               %
%%%%%%%%%%%%%%%%%%%%%%%%%%%%%%%%%%%%
\usepackage{setspace} % 1.5 Zeilenabstand
\usepackage[utf8]{inputenc} % Ermöglicht die Eingabe von deutschen Sonderzeichen im Quelltext
\usepackage[T1]{fontenc} % Europäische Kodierung der Ausgabeschrift
\usepackage{lmodern} % Lädt die Schriftart Latin Modern, die universell skalierbar ist
\usepackage[ngerman]{babel} % Silbentrennung nach neuer deutscher Rechtschreibung
\usepackage{amsmath} % Mathematik-Paket der American Mathematical Society
\usepackage{graphicx} % Erlaubt das Einfügen von Bildern per \includegraphics{}
\usepackage[style=alphabetic, backend=biber]{biblatex} %Verwendung von Biblatex zur Verwaltung der Literatur
\usepackage{csquotes} 
\usepackage[
  colorlinks,        % Links ohne Umrandungen in zu wählender Farbe
  linkcolor=black,   % Farbe interner Verweise
  filecolor=black,   % Farbe externer Verweise
  citecolor=black    % Farbe von Zitaten
]{hyperref}
\usepackage{float}
%\usepackage{fancyhdr} % neue Kopfzeilen mit fancypaket
\usepackage{geometry} % Packet für Seitenrandabständex und Einstellung für Seitenränder
\usepackage{listings} % Um Programmiercode anzuzeigen
\usepackage{xcolor}
\usepackage{color}
\usepackage{scrhack}

%%%%%%%%%%%%%%%%%%%%%%%%%%%%%%%%%%%%
%           /Packages              %
%%%%%%%%%%%%%%%%%%%%%%%%%%%%%%%%%%%%

% Seitenränder definieren
\geometry{left=3.5cm, right=2.5cm, headsep=1.5cm}

%%%%%%%%%%%%%%%%%%%%%%%%%%%%%%%%%%%%
%           Kopfzeile              %
%%%%%%%%%%%%%%%%%%%%%%%%%%%%%%%%%%%%
\usepackage[
  automark,
  autooneside=false,% <- needed if you want to use \leftmark and \rightmark in a onesided document
  headsepline
]{scrlayer-scrpage}
\clearpairofpagestyles
\ihead{\leftmark}
\ohead{\pagemark}
%%%%%%%%%%%%%%%%%%%%%%%%%%%%%%%%%%%%
%           /Kopfzeile             %
%%%%%%%%%%%%%%%%%%%%%%%%%%%%%%%%%%%%

%%%%%%%%%%%%%%%%%%%%%%%%%%%%%%%%%%%%
%        Quellcode einbinden       %
%%%%%%%%%%%%%%%%%%%%%%%%%%%%%%%%%%%%

\renewcommand{\lstlistingname}{Quellcode}% Listing -> Algorithm
\renewcommand{\lstlistlistingname}{\lstlistingname Verzeichnis}% List of Listings -> List of Algorithms

\definecolor{codegreen}{rgb}{0,0.6,0}
\definecolor{codegray}{rgb}{0.5,0.5,0.5}
\definecolor{codepurple}{rgb}{0.58,0,0.82}
\definecolor{backcolour}{rgb}{0.95,0.95,0.92}

\lstdefinestyle{mystyle}{
    language=[Sharp]C,
    backgroundcolor=\color{backcolour},   
    commentstyle=\color{codegreen},
    keywordstyle=\color{blue},
    numberstyle=\tiny\color{codegray},
    stringstyle=\color{codegreen},
    basicstyle=\ttfamily\footnotesize,
    breakatwhitespace=false,         
    breaklines=true,                 
    captionpos=b,                    
    keepspaces=true,                 
    numbers=left,                    
    numbersep=5pt,                  
    showspaces=false,                
    showstringspaces=false,
    showtabs=false,                  
    tabsize=2,
    morekeywords={  abstract, event, new, struct,
                      as, explicit, null, switch,
                      base, extern, object, this,
                      bool, false, operator, throw,
                      break, finally, out, true,
                      byte, fixed, override, try,
                      case, float, params, typeof,
                      catch, for, private, uint,
                      char, foreach, protected, ulong,
                      checked, goto, public, unchecked,
                      class, if, readonly, unsafe,
                      const, implicit, ref, ushort,
                      continue, in, return, using,
                      decimal, int, sbyte, virtual,
                      default, interface, sealed, volatile,
                      delegate, internal, short, void,
                      do, is, sizeof, while,
                      double, lock, stackalloc,
                      else, long, static,
                      enum, namespace, string, GeneticAlgorithmHarmonizer, var, final, String, LOG},
}
\lstset{style=mystyle}
%%%%%%%%%%%%%%%%%%%%%%%%%%%%%%%%%%%%
%       /Quellcode einbinden       %
%%%%%%%%%%%%%%%%%%%%%%%%%%%%%%%%%%%%