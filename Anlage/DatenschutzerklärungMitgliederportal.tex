\section{Bereitstellung des Mitgliederportals und Erstellung von Logfiles}
\label{cftPortaleDatenschutzerklaerung}
Bei jedem Aufruf des Mitgliederportals erfasst unser System automatisiert Daten und Informationen vom Computersystem des aufrufenden Rechners. Dabei werden folgende Daten erhoben:

\begin{itemize}
    \item die IP-Adresse des anfordernden Rechners (oder eines Proxyservers);
    \item Datum und Uhrzeit der Anforderung;
    \item der verwendete Login-Name
    \item vom anfordernden Rechner genutzten Browsertyp und Betriebssystem sowie deren Versionsnummern;
    \item vom anfordernden Rechner gewünschte Zugriffsmethode/Funktion;
    \item vom anfordernden Rechner übermittelte Eingabewerte (z.B. die Zieldatei);
    \item Zugriffstatus des Webservers (z.B. Datei übertragen, Datei nicht gefunden, Kommando nicht ausgeführt);
    \item der Name der angeforderten Datei;
    \item die URL (Internetadresse), von der aus die Datei angefordert/die gewünschte Funktion veranlasst wurde.
\end{itemize}
 
Im Internet benötigt jedes Gerät zur Übertragung von Daten eine eindeutige Adresse, die sogenannte IP-Adresse. Die zumindest kurzzeitige Speicherung der IP-Adresse ist aufgrund der Funktionsweise des Internets technisch erforderlich. \\
 
Die vorübergehende Speicherung der IP-Adresse durch das System ist notwendig, um eine Auslieferung des Mitgliederportals an den Rechner des Nutzers zu ermöglichen. 
Des Weiteren werden die Daten für statistische Auswertungen zur Optimierung der Dienste und für sicherheitstechnische Überprüfungen genutzt. 
Eine Speicherung dieser Daten zusammen mit anderen personenbezogenen Daten des Nutzers findet nicht statt. \\
 
Personenbezogene Daten für statistische Auswertungen werden unmittelbar nach Ihrer Erhebung anonymisiert und anonymisiert aufbewahrt und weiterverarbeitet. Die Daten für sicherheitstechnische Überprüfungen werden nach 14 Tagen gelöscht. \\
 
Rechtsgrundlage für die vorübergehende Speicherung der Daten ist Art. 6 Abs. 1 lit. f) DSGVO.
 
