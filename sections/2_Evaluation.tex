\chapter{Evaluation von Log-Management-Tools}
In der vorherigen Projektarbeit wurden für das CFT Portale Richtlinien definiert, die in dieser Bachelorarbeit praktisch umgesetzt werden sollen. 
Eine dieser Richtlinien war die Nutzung von einem zentralisierten Logging-Server mithilfe des Elastic Stacks. 
Jedoch wurden in der Projektarbeit keine weiteren Tools herangezogen, um zu prüfen, ob der Elastic Stack die beste Alternative ist. \\
In diesem Kapitel werden unterschiedliche Tools, die für das Log-Management genutzt werden können, evaluiert. 
Das Ziel dieser Evaluation ist es, zu prüfen, ob es eine bessere Alternative für eine zentralisierte Logging-Lösung gibt, als den Elastic Stack.
Dafür werden Tools evaluiert, die den kompletten Elastic Stack ersetzen können, aber auch Tools, die einzelne Komponenten austauschen können. \\
Das CFT Portale wäre in der Lage, weitere Kosten für ein Tool auf sich zu nehmen, sollte es dem Team die Arbeit erleichtern können. 
Daher werden Open-Source- und lizenzpflichtige Tools in dieser Evaluation betrachtet.
Sollten jedoch zwei Tools gleichermaßen die Anforderungen erfüllen und eines der Tools kostenlos sein, dann wird sich für das kostenlose Tool entschieden, um Kosten zu sparen. \\
Damit eine Evaluation erfolgen kann, müssen Anforderungen aufgestellt werden. 
Die Anforderungen sollen dabei helfen, eine Entscheidung bezüglich der Tools treffen zu können. 
Tools, die diese Anforderungen nicht erfüllen können, werden nicht weiter betrachtet.
Im nächsten Abschnitt werden die Anforderungen definiert.

\section{Anforderungen an die Log-Management-Tools}
\label{kap:anforderungenTools}
In diesem Abschnitt werden Anforderungen für die zentralisierten Logging-Tools definiert. 
Die Anforderungen helfen bei der Entscheidung, ein passendes Tool für das CFT Portale auszuwählen. 
Daher wurden in Absprache mit dem Team einige Anforderungen definiert, die das zukünftige Tool haben sollte. 
Für das CFT Portale spielt Wartung eine wichtige Rolle, daher werden sich einige Anforderungen auf den Wartungsaufwand beziehen. \\

Da das Netzwerk der KVWL sicherheitstechnisch stark abgeschirmt ist, kommt eine Cloud-Lösung nicht in Frage. 
Das bedeutet, dass das Tool eine Lösung bieten muss, die auf den Servern der KVWL läuft. \\
Durch die Menge an Anwendungen, die das CFT Portale betreuen muss, ist es wichtig, dass das Tool die einzelnen Logs entweder von den unterschiedlichen Maschinen selbst einsammeln kann oder das Senden der Logs möglich ist.
Das Installieren weiterer Log-Agenten, die für das Senden der Logs zuständig sind, sollten vermieden werden, da sonst weiterer Wartungsaufwand entstehen würde. \\
Im CFT Portale werden keine eigenen Datenbanken betreut.
Alle notwendigen Daten werden von anderen Teams zur Verfügung gestellt. 
Damit das CFT Portale keine weiteren Aufgaben erfüllen muss, fordert das CFT Portale, dass das Log-Management-Tool keine Logs in einer Datenbank speichert. \\ \\
Ein wichtiger Punkt ist die Analyse und Anzeige von Logs.
Das heißt, es soll eine Oberfläche vorhanden sein, die intuitiv benutzt werden kann, um die Logs anzusehen und zu analysieren. 
Jedoch sollte in dem Tool auch die Möglichkeit bestehen, Logs zu filtern und zu durchsuchen. \\
Im CFT Portale sind Linux- und Windows-Server im Betrieb. 
Jedoch möchte das Team das Tool gerne auf einem Linux-Server installieren, da dort das Updaten von neuen Versionen einfacher funktioniert. \\

Diese Vorgaben wurden im gemeinsamen Gespräch mit den Entwicklern des CFT Portale definiert. 
Mithilfe dieser Vorgaben soll ein passendes Log-Management-Tool für das CFT Portale ausgesucht werden. 
Die Anforderungen werden nachfolgend kurz zusammenfassend aufgelistet:

\begin{itemize}
    \item Selbsverwaltete Lösung (keine Cloud-Lösung)
    \item Einsammeln von Logs aus unterschiedlichen Anwendungen 
    \item Speicherung von Logs ohne externe Datenbank
    \item Anzeige und Analyse von Logs 
    \item Filtern und Durchsuchen von Logs 
    \item Installation auf Linux-Server 
\end{itemize}

\section{Log-Management-Tools}

In der Projektarbeit wurde der Elastic Stack in seiner Funktionsweise und dessen Möglichkeiten detailliert vorgestellt. 
Jedoch wird in diesem Kapitel auf die wichtigsten Inhalte des Elastic Stacks eingegangen, um eine Bewertung der Tools zu ermöglichen.
Alle Tools werden hier mit all ihren Vor- und Nachteilen vorgestellt.
Eine Bewertung der Tools wird in Kapitel \ref{kap:bewertungTools} durchgeführt.
Bevor die Tools evaluiert werden können, musste eine Vorauswahl getroffen werden. 
Dafür wurden viele Tools betrachtet und nach den Anforderungen in Kapitel \ref{kap:anforderungenTools} ausgewählt. 


\subsection{Elastic Stack}
% ALLGEMEIN
Der Elastic Stack ist eine Sammlung von Tools, die unterschiedliche Aufgaben erledigen. 
Die Kernprodukte sind Elasticsearch, Kibana, Logstash und Beat. 
Mit den Tools ist es möglich, zuverlässig Daten aus vielen unterschiedlichen Quellen zu erfassen und anschließend zu analysieren und visualisieren. \\
Mit der Komponente Elasticsearch ist das effiziente Suchen in großen Datenmengen möglich. \cite{elasticElasticStackElasticsearch}
In einem Ranking von Suchmaschinen wurde Elasticsearch als erster Platz ausgezeichnet und beweist damit, dass Elasticsearch schnell und effizient mit großen Datenmengen umgehen kann. \cite{db-enginesDBEnginesRanking} \\
\newpage
Die Aufgaben der einzelnen Tools sind:
\begin{itemize}
    \item Elasticsearch - Speichern und Suchen
    \item Kibana - Analyse und Anzeige 
    \item Logstash - Parsen und Weiterleiten
    \item Beat - Senden von Daten 
\end{itemize}

% VORAUSSETZUNGEN
Die Vorraussetzungen für die Nutzung des Elastic Stack sind niedrig. 
Denn die einzelnen Komponenten des Elastic Stacks können auf den aktuell vorhanden Betriebssystemen installiert werden. 
Zusätzlich zu den Komponenten des Elastic Stacks wird nur noch eine Installation von Java benötigt.
Java wird jedoch automatisch bei der Installation von Elasticsearch mitinstalliert. 
Elasticsearch wurde in Java geschrieben und muss daher in einer JVM (Java virtual machine) laufen. \cite{elasticElasticStackElasticsearch} \\

% KOSTEN
Die Nutzung des Elastic Stacks ist grundsätzlich kostenlos und ist zu einem großen Teil Open Source.
Es ist möglich den Quellcode des Elastic Stack in öffentlichen Repositories betrachtet. 
Der Elastic Stack kann als Open Source Version oder als Basic Version installiert werden.  
Die Open Source Version ist in den Features jedoch stark eingeschränkt. 
Wichtige Features, wie Sicherheitsmechanismen sind nicht nutzbar. 
Daher ist es ratsam die ebenfalls kostenlose Basic Version mit mehr Features zu nutzen, die nicht Open Source ist.
Zusätzlich zu den kostenlosen Versionen ist es möglich, ein kostenpflichtiges Abonnement abzuschließen um zusätzliche Features zu erhalten. 
Die wichtigsten Features, die durch das kostenpflichtige Abonnement zugänglich werden, sind Machine Learning und der Support. \\
Elastic Stack bietet zu der selbsverwalteten Lösung zusätzlich eine Cloud-Lösung die ohne Installation genutzt werden kann. 
Der Vorteil der Cloud-Lösung liegt im Wartungsaufwand, denn bei der Cloud-Lösung ist kein Installieren und Upgraden notwendig. 
Da die Anforderungen des Teams eine Cloud-Lösung ausschließen, wird die Cloud-Lösung nicht weiter beachtet. \cite{elasticOffiziellePreisinformationenElasticsearch} 

\newpage
\subsection{Graylog}
%ALLGEMEIN
Graylog ist ein Open Source Log-Management-Tool. 
Dessen Motto ist:
\begin{quote}
    \glqq  less cost, more performance\grqq \cite{graylogIndustryLeadingLog}
\end{quote}

Das Tool setzt auf Performance.
Die Funktionalitäten des Tools beziehen sich auf das Sammeln, Verbessern, Speichern und der Analyse von Logs. 
In Graylog kann man eigene Dashboards erstellen und individuell anpassen. 
Das Dashboard wird mithilfe von Suchabfragen definiert.
Damit nicht jeder Mitarbeiter sich mit den Suchabfragen beschäftigen muss, können die Dashboards untereinander geteilt werden. 
Graylog bietet zusätzlich vordefinierte Dashboards an, die genutzt werden können. \\
Mithilfe von Graylog können große Daten an Logs gespeichert werden.
Daher ist die Suche in großen Datenmengen essenziell. 
In Graylog werden die Logs beim Speichern indiziert, um eine effiziente Suche zu ermöglichen. 
Die Daten werden beim Speichern geprüft. 
Bei der Prüfung wird die Struktur genauer untersucht, um festzustellen, ob die Struktur in Ordnung ist.
Wenn die Struktur nicht in Ordnung ist, wird sie verbessert. \\
Die Architektur von Graylog ermöglicht eine multi-threaded Suche. 
Die Suche in Graylog ist dabei intuitiv aufgebaut. 
Einfache boolesche Operationen werden für die Suche genutzt. 
Die dazu benötigten Felder werden durch Klicken ausgewählt.
Damit muss keine neue Syntax erlernt werden und kann von unausgebildetem Personal genutzt werden. \cite{graylogIndustryLeadingLog} \\

% KOSTENSTRUKTUR
Wenn mehr benötigt wird, als die Open-Source-Version anbietet, dann kann Graylog als Enterprise-Variante gekauft werden. 
Bei der Enterprise-Variante werden Support und zusätzliche Funktionen angeboten.
Der genaue Vergleich der beiden Varianten kann in Abbildung \ref{fig:vergleichGraylogOSvsE} betrachtet werden.
Zu dem Support gehört Hilfe zu allen Graylog bezogenen Fragen, jedoch bietet Graylog zusätzlich Support für Elasticsearch, MongoDB und Oracle Java SE 8 (oder OpenJDK 8).
Sie bieten diesen Support an, weil Graylog diese Produkte benötigt, um in Betrieb genommen zu werden. 
Das bedeutet, dass diese Produkte zusätzlich auf der zu installierenden Maschine installiert werden müssen.  \\

Die Graylog Enterprise Variante kann kostenlos genutzt werden, solange die Menge an Logs die gespeichert werden, unter 5 GB/Tag bleiben.
Für Datenmenge bis zu 5 GB/Tag ist es daher ratsam auf die Enterprise Variante zu setzen. \cite{graylogGraylogOpenSource} \\
\begin{figure}[H]
    \center 
    \includegraphics[width=1\textwidth]{pictures/graylogVergleichOS-Enterprise.PNG} 
    \caption{Vergleich Graylog Open Source vs. Enterprise \cite{graylogGraylogOpenSource}}
    \label{fig:vergleichGraylogOSvsE}
\end{figure}

% VORRAUSSETZUNGEN
Graylog benötigt einige Systemvorraussetzungen, um installiert werden zu können. 
Zum Beispiel kann Graylog nur auf Linux-basierten Systemen installiert werden. 
Zusätzlich zu dem Betriebssystem muss noch Software installiert werden. 
Die zu installierende Software ist: \cite{graylogInstallingGraylogGraylog}
\begin{itemize}
    \item Elasticsearch 6.8+ oder 7
    \item MongoDB 3.6, 4.0 oder 4.2
    \item Oracle Java SE 8 (OpenJDK 8 funktioniert auch)
\end{itemize} 

An der Auflistung wird deutlich, dass spezielle Versionen der Tools installiert sein müssen, damit Graylog funktionieren kann. 
Dies kann bei der Wartung zu Problemen führen: 
Es muss beim Updaten auf neuere Versionen darauf geachtet werden, dass alle Versionen miteinander kompatibel sind.
Graylog bietet die Möglichkeit an, die Installation mithilfe von Docker oder dem Open Virtualization Format (OVA) durchzuführen. 
Das kann den Aufwand beim Updaten vereinfachen. 
Jedoch mit Installation auf Linux-Server ohne Docker oder OVAs besteht weiterhin das Problem der Kompatibilität der einzelnen Versionen miteinander. \\
Abschließend können die folgenden Vor- und Nachteile für Graylog zusammengefasst werden:
\begin{itemize}
    \item Vorteile: 
    \begin{itemize}
        \item Open Source (Enterprise auch möglich)
        \item Sehr performant durch multi-threaded Suche und Indizierung
        \item Speichern großer Datenmengen möglich
    \end{itemize}
     
    \item Nachteile: 
    \begin{itemize}
        \item Abhängigkeiten von externer Software
        \item Installation nur auf Linux
    \end{itemize}
\end{itemize}

\subsection{Sematext}
% ALLGEMEIN
Sematext ist ein kostenpflichtiges Log-Management- und Infrastructure-Monitoring-Tool. 
Das Tool wird als SaaS (Software as a Service) angeboten, jedoch gibt es zusätzlich eine Enterprise-Variante, die das Ausführen innerhalb der eigenen Infrastruktur ermöglicht. 
Das bedeutet, dass eine Kopie der Cloud Version von Sematext auf der eigenen Infrastruktur läuft. 
Sematext unterstützt im Vergleich zu anderen Tools nicht nur das Log-Management, sondern auch diese Funktionen: 
\begin{itemize}
    \item Infrastructure monitoring
    \item Application Performance Management (APM)
    \item Log Management
    \item Real User Monitoring 
\end{itemize}

Sematext nutzt für das Log-Management Elasticsearch und Kibana. 
Das bedeutet, beim Kauf von Sematext erhält man eine fertige Lösung des Elastic Stack. 
Der Vorteil besteht hier in der Wartung, die von Sematext selbst übernommen wird. 
Ein Vor- und Nachteil besteht beim Log-Shipper, der zusätzlich noch installiert werden muss. 
Dieser wird nicht von Sematext bestimmt. 
Daher kann flexibel entschieden werden, welcher Log-Shipper am besten geeignet ist. 
Jedoch kann dadurch auch nicht sichergestellt werden, dass der Log-Shipper mit dem Tool gut funktioniert. 
Mithilfe der Filter-Funktion können mit Sematext Benachrichtigungen konfiguriert werden. \cite{sematextSematextEnterpriseLog} \\
% VORAUSSETZUNGEN
Damit Sematext in der eigenen Infrastruktur betrieben werden kann, muss folgende Software installiert sein: 
\begin{itemize}
    \item Docker 
    \item Kubernetes 
    \item Helm
\end{itemize}

% INSTALLATION 
Sematext Enterprise wird als Helm Chart angeboten. 
Helm ist der Paketmanager für Kubernetes. Ein Chart ist eine Menge von Dateien, die zugehörige Kubernetes-Ressourcen beschreibt. 
Das Helm Chart beinhaltet alles, was nötig ist, um Sematext Enterprise nutzen zu können. 
Da Sematext Enterprise, nur als Helm Chart angeboten wird, muss die Installation über Kubernetes erfolgen. \cite{sematextSematextEnterpriseOverview} \\
Anschließend eine Auflistung der Vor- und Nachteile von Sematext:

\begin{itemize}
    \item Vorteile:
    \begin{itemize}
        \item Support und Wartung
        \item Benachrichtigungen
    \end{itemize} 
    \item Nachteile: 
     \begin{itemize}
        \item Kostenpflichtig
        \item Kein eigener Log-Shipper
        \item Nur eine Installation mit Docker möglich
    \end{itemize}
\end{itemize}


\subsection{Fluentd}
% ALLGEMEIN
Fluentd ist ein Open Source Data Collector, der es Anwendern ermöglichen soll, alles zu loggen. 
Das heißt, Fluentd sammelt Daten und leitet sie in gewünschter Form weiter zum Ziel. 
Also ist Fluentd kein Tool, um zentralisiertes Logging zu ermöglichen, sondern nur ein Tool, das beim zentralisierten Logging behilflich ist. 
Mithilfe von Fluentd können Datenströme einheitlich und verständlich ablaufen. 
Ein Beispiel dafür ist in Abbildung \ref{fig:fluentdVorherNachher} zu sehen.
Die Datenströme im \glqq Before Fluentd\grqq{} Teil sind chaotisch und unstrukturiert, wobei mit der Nutzung von Fluentd die Datenströme einheitlich über Fluentd laufen. \Cite{fluentdFluentdOpenSource} \\
Fluentd versucht, soweit es möglich ist, die erhaltenen Daten zu strukturieren. 
Dabei werden die Daten im JSON-Format gespeichert. 
Somit kann Fluentd die Daten einheitlich verarbeiten. 
Zum Verarbeiten gehören das Sammeln, Filtern, Puffern und die Weitergabe der Logs über verschiedene Quellen und Ziele hinweg. 
Für weitere Funktionalitäten können Lösungen von der Community genutzt werden. 
Das wird durch die Plugin-Architektur von Fluentd ermöglicht. 
Fluentd ist in einer Kombination von C und Ruby entwickelt worden. 
Deswegen benötigt es nur wenig Systemressourcen.
Die Standard-Version von Fluentd ohne weitere Komponenten benötigt ungefähr 30-40 MB Speicher. \cite{projectWhatFluentdFluentd}\\
Die größten Vorteile von Fluentd sind die Menge anverfügbaren Plugins und die Performance. 
Außerdem verbraucht Fluentd sehr wenig Speicher. 
Der größte Nachteil ist der Zwang zu strukturierten Daten. 
Dies bedeutet, dass es nicht so flexibel möglich ist, zu entscheiden, wie die Logs auszusehen haben. 
Fluentd formatiert die erhaltenen Daten immer im JSON-Format. 
Hier nochmal eine Auflistung der Vor- und Nachteile von Fluentd:

\begin{itemize}
    \item Vorteile:
    \begin{itemize}
        \item Open Source
        \item Große Auswahl an Plugins
        \item Performance
    \end{itemize} 
    \item Nachteile: 
     \begin{itemize}
        \item Zwang von strukturierten Daten
    \end{itemize}
\end{itemize}

\begin{figure}[H]
    \center 
    \includegraphics[width=1\textwidth]{pictures/Fluentd.PNG} 
    \caption{Fluentd Vergleich \cite{projectWhatFluentdFluentd}}
    \label{fig:fluentdVorherNachher}
\end{figure}


% https://sematext.com/blog/logstash-alternatives/
\section{Vergleich der Tools}
\label{kap:bewertungTools}

In diesem Kapitel erfolgt ein Vergleich der einzeln vorgestellten Tools mit dem Elastic Stack. 
So soll geprüft werden, welche Vor- und Nachteile die Tools im Vergleich zum Elastic Stack mitbringen. 
Anschließend soll mit Hilfe der in Kapitel \ref{kap:anforderungenTools} definierten Anforderungen entschieden werden, ob ein Tool für das CFT Portale besser geeignet ist als der Elastic Stack. 

\subsection{Graylog vs. Elastic Stack}

Graylog ist wie der Elastic Stack ein Open-Source-Tool, um zentralisiertes Logging zu ermöglichen. 
Dabei setzt Graylog bei der Suche auf Elasticsearch. 
Dadurch ist die Suche bei beiden Produkten identisch. \\
Damit Graylog gestartet werden kann, müssen Java, Elasticsearch und MongoDB vorher installiert werden. 
Graylog ist somit auf Drittanbieter-Software angewiesen. 
Beim Updaten muss darauf geachtet werden, dass die neue Graylog-Version mit den installierten Versionen der Drittanbieter-Software kompatibel ist. 
Das erzeugt einen geringfügigen Anstieg des Wartungsaufwandes. 
Jedoch ist das Updaten beim Elastic Stack ebenfalls mit Aufwand verbunden. 
Obwohl der Elastic Stack nur als Drittanbieter-Software Java benötigt, ist die Installation dennoch aufwendig. 
Denn beim Updaten müssen alle Tools in einer bestimmten Reihenfolge installiert werden. \cite{elasticUpgradingElasticStack} 
\begin{enumerate}
    \item Elasticsearch Hadoop 
    \item Elasticsearch
    \item Kibana
    \item Logstash
    \item Beats
    \item APM Server
\end{enumerate}
Der Wartungsaufwand ist durch die bestimmte Reihenfolge der Installation eher aufwendiger als bei Graylog. \\
Graylog bietet keinen eigenen Log-Agenten, der sich um das Senden von Logdateien kümmert. 
Graylog empfiehlt den Log-Agenten des Elastic Stacks zu nutzen, Filebeat. \cite{graylogIngestFilesGraylog} \\

Beide Tools bieten ähnliche Funktionen an und erfüllen alle Anforderungen die vom CFT Portale definiert wurden. 
Daher ist es nicht möglich ein Tool für die Nutzung im Team auszuschließen. 
Da die Zeit für diese Bachelorarbeit begrenzt ist, wird in dieser Bachelorarbeit der Elastic Stack installiert und getestet. 
Sollte der Test nicht zufriedenstellend ausgehen, sollte in Anschluss der Graylog installiert und getestet werden. 


\subsection{Sematext vs. Elastic Stack}
Im Gegensatz zum Elastic Stack ist Sematext kein Open-Source-Tool.
Damit Sematext auf der eigenen Infrastruktur betrieben werden kann, muss Sematext Enterprise erworben werden. 
Durch den Erwerb der Sematext Enterprise Version, erhält man eine Kopie des Sematext Cloud Service, der auf der eigenen Infrastruktur installiert werden kann. 
Mit Sematext ist es anschließend möglich, \textit{Infrastructure Monitoring}, \textit{Application Performance Monitoring}, \textit{Log Management} und \textit{Real User Monitoring} durchzuführen. 
Im Hintergrund von Sematext läuft ein voll konfigurierter Elastic Stack, der alle Funktionalitäten ermöglicht. 
Jedoch erhält man durch den Erwerb der Enterprise Version mehr Funktionen als die kostenlose Version des Elastic Stacks. \\
Der größte Vorteil von Sematext ist die einfache und schnellere Konfiguration des Systems. 
Bei der Installation von Sematext besteht jedoch ein Problem, denn die Installation erfolgt über Kubernetes und Docker-Container. 
Kein Teammitglied des CFT Portale besitzt das nötige Wissen, um die Installation und Wartung zu übernehmen. 
Daher kommt die Nutzung von Sematext nicht infrage. 

\subsection{Fluentd vs. Logstash}

Da Fluentd kein Tool ist, um zentralisiertes Logging zu ermöglichen, sondern nur eine Middleware, die beim Senden der Logs behilflich sein kann, wird das Tool mit Logstash alleine verglichen. \\
Fluentd ermöglicht die Vereinheitlichung von Daten aus unterschiedlichen Quellen in ein JSON-Format.
Durch die Strukturierung können die Daten besser analysiert werden. 
Mit Logstash ist dies ebenfalls möglich. 
Jedoch kann Logstash zusätzlich die Daten transformieren.
Zum Beispiel können IP-Adressen in geographische Daten transformiert werden und personenbezogene Daten können komplett anonymisiert werden. \\
Logstash und Fluentd bieten beide eine hervorragende Möglichkeit an, Logdaten zu strukturieren und an die erforderlichen Ziele weiterzuleiten. 
Jedoch ist es für das CFT Portale nicht notwendig die Daten zu strukturieren. 
Durch die erstellten Logging-Richtlinien werden die gespeicherten Logs im JSON-Format gespeichert und müssen daher nur noch an die richtige stelle gesendet werden. 
Da die Logs nur von selbsterstellten Anwendungen verwaltet werden sollen und nicht von zusätzlichen Systemen, reicht ein einfach Log-Shipper aus. 
Dafür wird die Komponente Filebeat des Elastic Stacks genutzt. 


\section{Zusammenfassung}

Im Rahmen dieser Evaluation wurden drei Tools zusätzlich zum Elastic Stack genauer untersucht. 
Das Ziel der Evaluation war die Antwort auf die Frage: \glqq Ist der Elastic Stack die beste Wahl für das CFT Portale?\grqq{}
Anhand der Anforderungen wurden eine Vielzahl an Tools untersucht. 
Dabei wurden viele der Tools durch die Anforderungen direkt ausgeschlossen. 
Es wurden drei Tools gefunden, die zu den Anforderungen des CFT Portale am ehesten passten. 
Während der Evaluation wurden die Tools mit dem Elastic Stack verglichen. 
Bei der Evaluation ist aufgefallen, dass der Elastic Stack und Graylog die Anforderungen des CFT Portale am besten erfüllen. 
Jedoch ist es nicht möglich im Rahmen dieser Bachelorarbeit beide Tools zu installieren und zu testen. 
Daher wird nur der Elastic Stack installiert und getestet.
In Tabelle \ref{tab:vergleichLogTools} ist eine Auflistung der getesteten Tools mit den relevanten Anforderungen dargestellt. 

\begin{table}[H]
    \centering
    \resizebox{\textwidth}{!}{%
    \begin{tabular}{|l|l|l|l|l|}
    \hline
                                                                                   & \textbf{Elastic Stack}                                            & \textbf{Graylog}                                                           & \textbf{Sematext}                                                    & \textbf{Fluentd}                                                  \\ \hline
    \textbf{\begin{tabular}[c]{@{}l@{}}Selbstverwaltete\\ Lösung\end{tabular}}     & Ja                                                                & Ja                                                                         & Ja                                                                   & Ja                                                                \\ \hline
    \textbf{\begin{tabular}[c]{@{}l@{}}Einsammeln \\ von Logs\end{tabular}}        & Ja, Filebeat                                                      & \begin{tabular}[c]{@{}l@{}}Nein,\\ Filebeat \\ empfohlen\end{tabular}      & Nein                                                                 & Nein                                                              \\ \hline
    \textbf{\begin{tabular}[c]{@{}l@{}}Speicherung \\ ohne Datenbank\end{tabular}} & Ja                                                                & Ja                                                                         & Ja                                                                   & \begin{tabular}[c]{@{}l@{}}Kein \\ Speichern\end{tabular}         \\ \hline
    \textbf{\begin{tabular}[c]{@{}l@{}}Anzeige und \\ Analyse\end{tabular}}        & Ja                                                                & Ja                                                                         & Ja                                                                   & Nein                                                              \\ \hline
    \textbf{\begin{tabular}[c]{@{}l@{}}Filtern und \\ Duchsuchen\end{tabular}}     & Ja                                                                & Ja                                                                         & Ja                                                                   & Nein                                                              \\ \hline
    \textbf{Linux-Installation}                                                    & Ja                                                                & Ja                                                                         & Nein                                                                 & Ja                                                                \\ \hline
    \textbf{Open Source}                                                           & \begin{tabular}[c]{@{}l@{}}Ja, \\ eine Version\end{tabular}       & Ja                                                                         & Nein                                                                 & Ja                                                                \\ \hline
    \textbf{\begin{tabular}[c]{@{}l@{}}Drittanbieter\\ Software\end{tabular}}      & Java                                                              & \begin{tabular}[c]{@{}l@{}}MongoDB, \\ Java, \\ Elasticsearch\end{tabular} & \begin{tabular}[c]{@{}l@{}}Docker,\\ Kubernetes,\\ Helm\end{tabular} & Nein                                                              \\ \hline
    \textbf{Wartungsaufwand}                                                       & Mittel                                                            & Mittel                                                                     & Niedrig                                                              & Niedrig                                                           \\ \hline
    \textbf{Installation}                                                          & \begin{tabular}[c]{@{}l@{}}Linux,\\ Windows,\\ MacOS\end{tabular} & \begin{tabular}[c]{@{}l@{}}Linux,\\ Docker,\\ OVA\end{tabular}             & \begin{tabular}[c]{@{}l@{}}Docker,\\ Kubernetes\end{tabular}         & \begin{tabular}[c]{@{}l@{}}Linux,\\ Windows,\\ MacOS\end{tabular} \\ \hline
    \end{tabular}%
    }
    \caption{Vergleich der Log-Management-Tools}
    \label{tab:vergleichLogTools}
    \end{table}