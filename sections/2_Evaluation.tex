\chapter{Evaluation von Log-Management Tools}
In der vorherigen Projektarbeit wurden für das CFT Portale Richtlinien definiert, die in dieser Bachelorarbeit praktisch umgesetzt werden sollen. 
Eine dieser Richtlinien war die Nutzung von einem zentralisierten Logging-Server mithilfe des Elastic Stack. 
Jedoch wurden in der Projektarbeit keine weiteren Tools herangezogen, um zu prüfen, ob der Elastic Stack die beste Alternative ist. \\
In diesem Kapitel werden unterschiedliche Tools, die für das Log-Management genutzt werden können, evaluiert. 
Das Ziel dieser Evaluation ist zu prüfen, ob es eine bessere Alternative für eine zentralisierte Logging Lösung gibt, als den Elastic Stack.
Dafür werden Tools evaluiert, die den kompletten Elastic Stack ersetzen können, aber auch Tools die einzelne Komponenten austauschen können. \\
Das CFT Portale wäre in der Lage weitere Kosten für ein Tool auf sich zu nehmen, sollte es dem Team die Arbeit erleichtern können. 
Daher werden Open-Source und Lizenzpflichtige Tools in dieser Evaluation betrachtet.
Sollten jedoch zwei Tools gleichermaßen die Anforderungen erfüllen und eines der Tools Open-Source sein, dann wird sich für das Open-Source Tool entschieden, um kosten zu sparen. \\
Damit eine Evaluation erfolgen kann, müssen Anforderungen aufgestellt werden. 
Diese Anforderungen sollen dabei helfen eine Entscheidung bezüglich der Tools treffen zu können. 
Denn Tools die diese Anforderungen nicht erfüllen können, werden nicht weiter betrachtet.
Im nächsten Teil diesen Kapitels werden diese Anforderungen definiert. 

\section{Anforderungen an die Log-Management Tools}

\begin{itemize}
    \item Selbstorganisierte Lösung (Kein CLoud)
    \item Tool soll Logs von unterschiedlichen Anwendungen einsammeln können 
    \item Speichern von Logs 
    \item Analyse(Anzeigen) von Logs an einer Stelle möglich 
    \item Filtern und Dursuchen von Logs 
    \item Im Bestenfall auf Linux(Redhat) installierbar, Windows auch ok
\end{itemize}

\section{Log-Management Tools}

// TODO: Kurz erläutern Elastic Stack und warum der nicht genauer erklärt wird.
\subsection{Graylog}
\subsection{Loggly}
\subsection{Fluentd}
\subsection{Splunk}
\subsection{LogDNA}

