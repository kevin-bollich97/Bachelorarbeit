\chapter{Logging Allgemein}
\label{loggingAllgemein}
Obwohl Logging eine sehr mächtige und hilfreiche Programmiermethode ist, wird sie in der Entwicklung und Wartung von Softwaresystemen stark unterschätzt.
Während der Laufzeit, können die durch Logging gespeicherten Laufzeit Informationen Entwicklern helfen, einen Fehler zu reproduzieren.
Durch die Reproduktion kann die Ursache eines Fehler erkannt und anschließend behoben werden.
Die Reproduktion von Fehlern ist nicht der einzige Anwendungsfall von Logging. 
Denn Logging kann in unterschiedlichen Bereichen der Softwaretechnik behilflich sein.
Zu den Bereichen gehören unter anderem: \cite{fuWhereDevelopersLog2014}
\begin{itemize}
    \item Anomaly Detection
    \item Error Debugging
    \item Performance Diagnosis 
    \item Workload Modeling
    \item System Behavior Understanding
\end{itemize} 
Während des Entwicklungsprozesses müssen Entwickler sich überlegen, welche Informationen in ihrer Software relevant sind, um diese während der Laufzeit zu speichern. 
Eine zu niedrige oder zu hohe Informationsdichte beim Logging kann mehr Probleme verursachen als beheben.
Die möglichen Probleme werden in Kapitel \ref{zuVielLoggen} genauer erläutert. \cite{fuWhereDevelopersLog2014} \\

Die Autoren \citeauthor{davidekovaSoftwareApplicationLogging2016} schreiben in ihrem Artikel \citetitle{davidekovaSoftwareApplicationLogging2016}:
\begin{quote}
    \glqq [\dots] logging can be seen as
    knowledge transfer from the application to the developer in
    form of logged data in the log file.\grqq \cite{davidekovaSoftwareApplicationLogging2016}
\end{quote}
Damit beschreiben sie Logging als eine Art Kommunikation zwischen Software und Entwicklern.
Die Logdateien sind dabei das Medium mit dem kommuniziert wird. 
Ohne Informationen ist es für Entwickler schwierig, einen Fehler zu reproduzieren und diesen dann zu beheben. \cite{davidekovaSoftwareApplicationLogging2016} \\ 


Wie kann die Software mit dem Entwickler kommunizieren? \\
Logging-Frameworks ermöglichen die Kommunikation von der Software zum Entwickler.
Die Frameworks bieten Funktionen an, um an den gewünschten Stellen im Quellcode, eine Meldung hinterlassen zu können. 
Wo diese Meldungen angezeigt oder gespeichert werden, muss in den jeweiligen Konfigurationsdateien definiert werden.
Um genauer zu definieren, welche Meldungen unter welchen Umständen aufgerufen werden sollen, gibt es sogenannte Log-Level.
Die Log-Level beschreiben die Kategorien, in denen die Meldungen jeweils zugeordnet werden. 
In Kapitel \ref{loglevel} werden Log-Level genauer beschrieben.

\section{Welche Probleme können beim Loggen zu vieler oder zu weniger Daten entstehen?}
\label{zuVielLoggen}
Damit Logging effizient genutzt werden kann, muss klar definiert werden, welche Softwareaktionen kritisch behandelt werden müssen.
Dabei ist es wichtig, dass genug Daten geloggt werden, aber keine unnötigen oder trivialen Informationen. 
Durch das Loggen von irrelevanten Informationen können die hilfreichen und wichtigen Informationen in der Menge von Daten verloren gehen. 
Um diesem Problem entgegen zu wirken, sollten alle Prozesse der Software vollständig von allen beteiligten Personen verstanden werden.  \cite{fuWhereDevelopersLog2014} \\

Jede Log-Operation verbraucht CPU-Leistung und Speicherplatz. 
Daher kann das massenhafte Loggen von Informationen die Performance der Software beeinträchtigen. \\
Generell muss entschieden werden, wie lange Logdateien erhalten bleiben sollen. 
Da geht es nicht nur um den genutzten Speicherplatz, sondern auch um den Datenschutz. 
Dieser Aspekt wird in Kapitel \ref{datenschutz} diskutiert. \\

Da Logging generell Entwicklungs- und Wartungszeit benötigt, bewirkt das zu viele Loggen, dass die Kosten für Entwicklung und Wartung steigen. 
\cite{fuWhereDevelopersLog2014} \\

Bis jetzt wurden nur die Probleme des zu vielen Loggings erwähnt. 
Jedoch ist es auch Wichtig, genug Informationen zu Loggen. 
Beim zu wenig Loggen können essenzielle Laufzeitinformationen fehlen. 
Ohne diese Informationen kann die Fehler-Analyse nicht effizient oder gar nicht durchgeführt werden. \cite{fuWhereDevelopersLog2014}


\section{Welche Informationen sollten geloggt werden? }
\label{welcheInformationen}
Damit Entwickler einen Nutzen aus den entstandenen Logdateien ziehen können, müssen bestimmte Informationen enthalten sein. 
In der Praxis beinhalten Logs oft zu wenig Informationen, um daraus herleiten zu können, wie ein Fehler entstanden ist. 

Entwickler müssen beim Identifizieren eines Fehlers den Softwarecode oft debuggen. 
Dies resultiert aus der mangelnden Qualität der zur Verfügung gestellten Logs.
Ein Log sollte die meisten Fragen beantworten können, auf die ein Entwickler bei der Fehlersuche stoßen kann. 
Zu den Fragen gehören: \cite{davidekovaSoftwareApplicationLogging2016}

\begin{itemize}
    \item \textbf{Was} ist passiert?
    \item \textbf{Wann} ist es passiert?
    \item \textbf{Wo} ist es passiert?
    \item \textbf{Wer} hat das Event ausgelöst?
    \item \textbf{Warum} ist es passiert?
    \item \textbf{Was} war die Ursache?
\end{itemize}

Jede dieser Fragen kann mithilfe einfacher Informationen beantworten werden. \\
Die erste Frage nach dem \textbf{was} wird durch den Typ des Logs beschrieben, dazu gehören zum Beispiel Warnung, Error oder Info. \\
\textbf{Wann} ein Log passiert ist, wird durch einen einfachen Timestamp beantwortet. \\
Bei der Frage nach dem \textbf{wo}, muss das ausführende System im Log beschrieben sein. Wenn bei der Kommunikation zweier Systeme ein Problem entstanden ist, dann muss beschrieben werden, welches das Quell-System und welches das Ziel-System war. \\ 
Um zu wissen, \textbf{wer} ein Event ausgelöst hat, kann der Username eingetragen werden. Jedoch muss dabei der Datenschutz beachtet werden.
Dieses wird in Kapitel \ref{datenschutz} beschrieben. \\
\textbf{Warum} ein Log geschrieben wurde, kann zum Beispiel sein: Falsches Passwort eingegeben.
Generell werden die entstandenen Exceptions zurück geliefert. \\
Die letzte Frage bezieht sich darauf, \textbf{was} das ausgelöste Event war. Da soll dann der Name einer Methode mitgeliefert werden. Dazu gehören zum Beispiel, Login, eine bestimmte Berechnung oder der Aufruf einer Datei. \cite{davidekovaSoftwareApplicationLogging2016}

\section{Log-Level}
\label{loglevel}
Mithilfe von Log-Leveln können Entwickler die Anzahl der gespeicherten Logs nach Anwendungsfall variieren. 
Somit kann ermöglicht werden, dass während des produktiven Betriebs der Software, nur relevante Informationen gesichert werden. 
Die Informationen, die für die Produktion unwichtig erscheinen, werden für den Entwicklungsprozess benötigt. 
Daher werden unterschiedliche Log-Level angeboten, die nach Anwendungsfall ausgetauscht werden können. 
Die sechs relevanten Log-Level sind \textit{Trace}, \textit{Debug}, \textit{Info}, \textit{Warn}, \textit{Error} und \textit{Fatal}. 
In Abbildung \ref{fig:loglevel} ist die Hierarchie der einzelnen Log-Level untereinander dargestellt. \cite{liWhichLogLevel2016}
\begin{figure}[H]
   
    \center 
    \includegraphics[scale=0.3]{pictures/LogLevel.png}   
    \caption{Log-Level Hierarchie}
    \label{fig:loglevel}
\end{figure}

\textbf{Trace} ist das ausführlichste der Log-Level. 
Bei Trace werden alle Details, die das Verhalten der Software repräsentieren, geloggt. 
Die erhaltenen Informationen können für Administratoren und Entwickler relevant sein. \\
In \textbf{Debug} wird nicht wie in Trace jedes Detail geloggt, aber vom Umfang ist es ähnlich. 
Die Informationen aus dem Log-Level Debug werden hauptsächlich von Entwicklern und Administratoren verwendet. \\
\textbf{Info} beinhaltet, wie der Name schon sagt, Informationen über den Ablauf. Zum Beispiel beschreibt es, welche Systeme gestartet sind oder auch welcher Datensatz neu in die Datenbank geschrieben wurde. \\
Wo Info nur informativ ist und keine Probleme andeutet, ist \textbf{Warn} ein Zeichen dafür, dass etwas nicht so abgelaufen ist, wie es sollte. Warn ist die letzte Stufe bevor ein Fehler entsteht. \\
Durch \textbf{Error} werden Fehler angezeigt, die keinen Systemabsturz verursacht haben. Dazu gehörten zum Beispiel, dass die Verbindung der Datenbank unterbrochen ist. Die Anwendung kann also noch funktionieren, es sollte jedoch recht schnell behoben werden. \\
\textbf{Fatal} beschreibt katastrophale Fehler, die einen Absturz der Software verursachen. \\




\section{Wo sollten Logdateien gespeichert werden? }
\label{kap:zentralLogging}

Logs können unterschiedlich angezeigt oder gespeichert werden. 
Nicht jeder Fehler bringt einen Systemabsturz oder eine Fehlermeldung an den Nutzer der Software mit sich.
Fehler können entstehen, ohne dass jemand diese auf Anhieb erkennen kann.
Für Entwickler und Administratoren ist es daher wichtig, Logdateien für längere Zeit zu sichern. \\ 
Die Logdateien werden in der Regel auf den Servern erstellt, auf denen die jeweilige Software läuft. 
Wenn ein Entwicklerteam mehr als eine Software warten und entwickeln muss, dann können die Applikationen auf unterschiedlichen Servern laufen. 
Wenn jede Software ihr eigens Logdokument auf dem eigenen Server speichert, dann kann dies unübersichtlich werden. 
Bei Software, die parallel auf zwei Servern läuft, kann es schwierig sein, den Pfad eines Prozess zurückzuverfolgen. 
Daher ist ein zentrales Logging essenziel für solche Anwendungsfälle. \cite{vainioImplementationCentralizedLogging2018} \\ 


Beim zentralisierten Loggen werden die Logs unterschiedlicher Systeme und Applikationen an einem zentralen Punkt zusammengeführt. 
Der zentrale Punkt kann ein Server sein auf dem eine Datenbank bereitgestellt ist.
Die Datenbank ermöglicht die Filterung und Sortierung der Logs, so dass eine schnellere und präzisere Fehlersuche möglich wird. \cite{vainioImplementationCentralizedLogging2018} \\


In Abbildung \ref{fig:centralLogging} wird ein Prozess des zentralen Loggings dargestellt.
Er zeigt, wie die Logs von der erzeugenden Maschine (log producing machine) bis zum zentralisierten Logging Server (centralized logging server) gelangen.\\
In der Umgebung, in der die Logs erzeugt werden, sind drei Komponenten vorhanden. 
Einmal die Anwendung, die als \textit{log producer} gekennzeichnet ist. 
Dann die produzierte Logdatei, deren Informationen weitergeleitet werden müssen.
Als letzte Komponente kommt die \textit{log transmission}, die für die Übertragung des Logs zuständig ist. 
Um Bandbreite zu sparen, sollten die Logs komprimiert werden. 
Da Logs nur aus Text bestehen ist dies effizient möglich. \cite{vainioImplementationCentralizedLogging2018} \\
Der centralized logging server in Abbildung \ref{fig:centralLogging} enthält genau wie die \textit{log producing machine} drei Komponenten. 
Der log collector empfängt die vom \textit{log transmission} gesendeten Logs und speichert sie in der Komponente \textit{log storage}.
Nach der Speicherung können die Logs noch von einer weiteren Komponente analysiert werden, um Berichte dadurch erstellen zu können. \cite{vainioImplementationCentralizedLogging2018}

\begin{figure}[H]
    \includegraphics[width=1\textwidth]{pictures/CentralLoggingProzess.PNG}   
    \caption{Prozess zentrales Logging \cite{vainioImplementationCentralizedLogging2018}}
    \label{fig:centralLogging}
\end{figure} 

Der in Abbildung \ref{fig:centralLogging} vorgestellte Prozess ist nicht für jeden Anwendungsfall relevant. 
In anderen Prozessen kann die Komponente \textit{log transmission} die Daten direkt in \textit{log storage} speichern. 
Da würde dann der \textit{log collector} nicht mehr benötigt werden. 
Daher muss der Prozess individuell bestimmt werden. \\ 

Bei Software, die ein stetiges Wachstum zeigt, kann das Ausgliedern von Log-Informationen auf einen zentralisierten Server sinnvoll sein. 
Da der Trend von Cloud-basierten Technologien immer mehr an Relevanz gewinnt, wird das Zentralisieren von Log-Informationen immer wichtiger. 
Unterschiedliche Komponenten eines Softwaresystems können auf viele Systeme verteilt sein (z.B. Microservice Architektur). 
Da kann eine zentralisierte Logging-Architektur behilflich sein, um ein Bugtracking mithilfe von Logfiles überhaupt erst zu  ermöglichen. 


\section{Logging-Frameworks}

Um das Logging zu vereinfachen, werden Logging-Frameworks angeboten. 
Die Vereinfachung wird durch standardisierte Funktionen und andere Features erreicht.  
Jedes Framework bietet unterschiedliche Log-Level, um die entstandenen Logs filtern zu können. 
In diesem Abschnitt werden drei Logging-Frameworks vorgestellt, die für das .Net Framework angeboten werden. 
Das .Net Framework ist hier relevant, weil das CFT Portale ausschließlich in .Net entwickelt. 
Die Frameworks sind: \textit{Log4Net}, \textit{NLog} und \textit{Serilog}


\subsection{NLog}
NLog ist eines der beliebtesten Logging-Frameworks für .Net. 
Die Version 1.0 erschien 2006 und steht unter aktiver Weiterentwicklung. \cite{NLogVsLog4net2018} \\ 
NLog letztes Update erfolgte am 18.05.2020 (Stand: 19.05.2020). \cite{nlogNLog} \\
Performance und die simple Konfiguration gehören zu den maßgebenden Vorteilen von NLog.
NLog bietet die Möglichkeit, strukturiertes Logging zu verwenden. 
Beim strukturierten Logging werden die Logs nicht im Text-Format gespeichert, sondern in einer strukturierten Form, wie z.B. im XML- oder JSON-Format. \\
Das Framework kann über den NuGet Package Manager von Microsoft runtergeladen werden. 
Um zu definieren, wo die Logs gespeichert oder hingeschickt werden sollen, muss dann die \textit{NLog.config} Datei bearbeitet werden. 
Diese Konfiguration kann jedoch auch im Code erfolgen und nicht in der Konfigurationsdatei.
Zu den Speichermöglichkeiten gehören: Logdatei, Mail, Konsole oder Datenbanken.
Um den Ort der Speicherung anzupassen, muss ausschließlich die Konfigurationsdatei verändert werden, das Kompilieren oder Ändern von Codeabschnitten ist nicht erforderlich.
Die von NLog unterstützen Log-Level sind: \textit{Trace}, \textit{Debug}, \textit{Info}, \textit{Warn}, \textit{Error} und \textit{Fatal}. \cite{swerskyLoggingBestPractices2018}\\

Hier ein Beispiel, wie Aufrufe von NLog im Code aussehen: 
\begin{lstlisting}[caption=Beispielhafter Aufruf von NLog]
    private static NLog.Logger logger = 
        NLog.LogManager.GetCurrentClassLogger();
    logger.Info("Hello {0}", "World");
    logger.Error(exception, "Something bad happened");
\end{lstlisting}

\subsection{Log4Net}
\label{log4net}
Log4Net ist eine Portierung des bekannten \textit{Log4J} Logging-Frameworks für Java in das .Net Umfeld.
Dies geschah 2001 und ist somit das älteste Logging-Frameworks für .Net. \cite{timmsNLogVsLog4net2018}\\
Die aktive Wartung wurde jedoch zum 01.04.2020 eingestellt. 
Das letzte durchgeführte Update ist vom 11.03.2017. \cite{apacheApacheLog4netApache} \\

Die Konfiguration von Log4Net erfolgt ähnlich wie bei NLog. 
In der Konfigurationsdatei von Log4Net wird beschrieben, wo die geloggten Daten hingesendet werden.
Log4Net nutzt so genannte Appender, um Logdaten an die unterschiedlichen Systeme zu schicken. 
Die von Log4Net unterstützten Log-Level entsprechen denen von NLog, mit der Ausnahme, dass Log4Net kein \textit{Trace} level besitzt. \cite{swerskyLoggingBestPractices2018}\\

Hier ein Beispiel wie Aufrufe von Log4Net im Code aussehen: 
\begin{lstlisting}[caption=Beispielhafter Aufruf von Log4Net]
    private static readonly ILog log = 
        LogManager.GetLogger(typeof(MyApp));
    log.Info("Starting application.");
    log.Debug("DoTheThing method returned X");
\end{lstlisting}


% https://raygun.com/blog/c-sharp-logging-best-practices/

\subsection{Serilog}
Serilog ist das neuste der drei Logging-Frameworks, es wurde erst 2013 veröffentlicht.
Der große unterschied von Serilog gegenüber NLog und Log4Net ist, dass es so designed wurde, dass strukturiertes Logging von Beginn an genutzt werden kann. \\
Bei der Konfiguration können beide Verfahren genutzt werden. 
Entweder erfolgt die Konfiguration im Programmcode oder in einer Konfigurationsdatei.
Die Konfiguration ist wie bei Log4Net und NLog sehr simpel. \cite{NLogVsLog4net2018}
\newpage

Hier ein Beispiel wie Aufrufe von Serilog im Code aussehen: 
\begin{lstlisting} [caption=Beispielhafter Aufruf von Serilog]
    Log.Logger = new LoggerConfiguration()
                .MinimumLevel.Debug()
                .WriteTo.Console()
                .WriteTo.File("logs\\my_log.log", 
                    rollingInterval: RollingInterval.Day)
                .CreateLogger();

    Log.Information("Hello, Serilog!");
    Log.Warning("Goodbye, Serilog.");
\end{lstlisting}

Das Loggen erfolgt ähnlich wie bei den beiden anderen Frameworks. 

% https://michaelscodingspot.com/logging-in-dotnet/
% https://stackify.com/nlog-vs-log4net-vs-serilog/

\section{Datenschutz}
\label{datenschutz}

Seit dem 25. Mai 2018 gilt die EU-Datenschutz-Grundverordnung (EU-DSGVO). 
Diese beinhaltet Richtlinien für den Schutz natürlicher Personen bei der Verarbeitung von personenbezogenen Daten. 
Seit der DSGVO müssen Websitebetreiber den Nutzer über die Verarbeitung ihrer personenbezogenen Daten informieren.
Hierzu soll eine Datenschutzerklärung  auf der Website vorhanden sein und alle nötigen Informationen liefern.
Da beim Logging personenbezogene Daten gesammelt werden, müssen bei der Verarbeitung und Speicherung der Daten die vom Europäischen Parlament bestimmten Richtlinien eingehalten werden. \cite{RichtlinieEU20162016} \\


Anhang \ref{cftPortaleDatenschutzerklaerung} zeigt die Datenschutzerklärung der KVWL, die in ihrem Mitgliederportal zur Verfügung gestellt wird.
In der Erklärung stehen alle Daten, die vom Portal erhoben werden. 
Diese Datenschutzerklärung bezieht sich auf die Daten, die in den Logfiles gespeichert werden und erklärt bestimmte IT-spezifische Begriffe. 
Wie in der DSGVO gefordert, wird auch der Grund der Speicherung erläutert. 
In der Erklärung wird auch, der DSGVO entsprechend, auf die Speicherdauer und die Löschung der Daten eingegangen. 


% Datenschutz anschneiden 
% Was muss beachtet werden 
% muss was beachtet werden? 

\section{Überblick}

In diesem Kapitel war das Ziel, ein generelles Verständnis vom Logging zu erhalten. 
Dabei wurden für die Projektarbeit relevante Technologien und Verfahren im Bezug aufs Logging vorgestellt. 
Dabei wurden grundlegende Fragen beantwortet. 
Zu Beginn des Kapitels wurden die Probleme erläutert, die beim Loggen zu vieler Daten entstehen können. 
Anschließend wurde darauf eingegangen, welche Informationen geloggt werden sollen. 
Ein wichtiger Punkt im Logging ist das Log-Level.
Um zu verstehen welche Log-Level es gibt und wie diese funktionieren, wurden diese erklärt und zur Veranschaulichung in einer Grafik dargestellt.
Das zentralisierte Logging ist ein wichtiger Bestandteil dieser Projektarbeit. 
Daher wurde in diesem Kapitel erklärt, was das zentrale Logging überhaupt ist und wie es funktioniert. 
Zum Schluss wurden noch Logging-Frameworks vorgestellt und deren Funktionsweise erläutert. 




