\chapter{Umsetzung der Richtlinien}

In diesem Kapitel werden die Richtlinien der vorher geschriebenen Projektarbeit praktisch an der Anwendung \textit{Vierteljahreserklärung} durchgeführt.
Dafür muss der Quellcode angepasst werden, sowie einzelne Konfiguration der Anwendung. 
Um das verständnis für die Anwendung zu erhalten, wird zu beginn erstmal beschrieben wofür die \textit{Vierteljahreserklärung} überhaupt genutzt wird und wie die Anwendung entwickelt wurde.

\section{Vierteljahreserklärung}
// Anfang was ist VJE? 

Die Provider hosted App \textit{Vierteljahreserklärung} ermöglicht die Generierung einer PDF für die Vierteljahreserklärung.
Dabei ist die Vierteljahreserklärung eine Erklärung, die von den Mitgliedern der Kassenärztlichen Vereinigung Westfalen-Lippe an die KVWL gesendet werden muss. 
Die Mitglieder müssen dabei unterschiedliche Angaben zu dem Quartal abgeben. 
Am Ende muss ein Quartal angegeben werden, für das die Vierteljahreserklärung erstellt werden soll. 
Nach der Angabe des Quartals erfolgt eine Überprüfung der Software, für welche BSNR das Mitglied berechtigt ist eine Vierteljahreserklärung abzugeben. 
Nach Auswahl der BSNR kann das PDF Dokument heruntergeladen werden. 
Die Provider hosted App kann in Abbildung \ref{fig:vje} genauer betrachtet werden. 

// BILD NOCHMAL AUF DER ARBEIT SCREENSHOTEN (GRÖßERER MONITOR UND HÖHERE AUFLÖSUNG)
\begin{figure}[H]
    \center 
    \includegraphics[width=1\textwidth]{pictures/VJE.PNG} 
    \caption{Vierteljahreserklärung }
    \label{fig:vje}
\end{figure}

// Ende wie ist es entwickelt



\section{Konfiguration von NLog}

\section{Aufbau der Logs}

\section{Abgleich der Richtlinien}

