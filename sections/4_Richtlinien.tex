\chapter{Umsetzung der Richtlinien}

In diesem Kapitel werden die Richtlinien der vorher geschriebenen Projektarbeit praktisch an der Anwendung \textit{Vierteljahreserklärung} durchgeführt.
Dafür muss der Quellcode, sowie einzelne Konfigurationen der Anwendung angepasst werden. 
Um das Verständnis für die Anwendung zu erhalten, wird zu Beginn beschrieben, wofür die \textit{Vierteljahreserklärung} überhaupt genutzt wird und wie die Anwendung entwickelt wurde.

\section{Vierteljahreserklärung}
\label{kap:vje}
Die Provider hosted App \textit{Vierteljahreserklärung} ermöglicht die PDF-Generierung der \textit{Erklärung zur Vierteljahresabrechnung}. 
Mit der \textit{Erklärung zu Vierteljahresabrechnung} bestätigt der Vertragsarzt seine abgerechneten Leistungen. 
Dies ist in den Abrechnungsrichtlinien der KVWL festgelegt. 
Der Ausschnitt der Richtlinien, der die Pflichten des Vertragsarztes bezüglich der \textit{Erklärung der Vierteljahresabrechnung} beschreibt, ist anschließend zitiert. 
Der Ausschnitt ist in §7 Absatz 1 zu finden:    

\begin{quote}
  \glqq Der Vertragsarzt hat gegenüber der KVWL auf dem der Abrechnung beizufügenden Vordruck „Erklärung zur Vierteljahresabrechnung“ (§ 35 Abs. 2 BMV-Ä)
  schriftlich zu bestätigen, dass die zur Abrechnung gestellten Leistungen unter Beachtung der Vorgaben nach § 2 Abs. 1 tatsächlich erbracht worden sind und die
  Abrechnung sachlich-rechnerisch richtig ist. Die Vorlage des unterschriebenen
  Vordrucks „Erklärung zur Vierteljahresabrechnung“ ist Abrechnungsvoraussetzung.\grqq{} \cite{kassenaerztlichevereinigungwestfalen-lippeAbrechnungsrichtlinienKassenaerztlichenVereinigung2015}
\end{quote}

\newpage

Mithilfe der Anwendung \textit{Vierteljahreserklärung} können die Angaben, die für die Vierteljahresabrechnung notwendig sind, eingetragen werden. 
Ein Screenshot der Anwendung ist in Abbildung \ref{fig:vje} abgebildet. 
Bevor das PDF generiert werden kann, ist die Eingabe des Quartals und der Betriebsstättennummer (BSNR) Voraussetzung. \\

Die BSNR ist eine neunstellige Zahl, die eine Zuordnung ärztlicher Leistungen zum Ort der Leistungerbringung ermöglicht. 
Die ersten beiden Ziffern stellen den KV-Landes- oder Bezirksstellenschlüssel dar. 
Die Ziffern drei bis neun werden von der Kassenärztlichen Vereinigung vergeben, in dessen Bereich die Betriebsstätte liegt. 
Bei Abrechnungen muss die BSNR mit angegeben werden. 
Dies ist in den Richtlinien der Kassenärztlichen Bundesvereinigung (KBV) festgelegt worden. 
Eine BSNR bekommt jede Betriebsstätte und Nebenbetriebsstätte, sowie Einrichtungen, die zur Teilnahme an der vertragsärztlichen Versorgung ermächtigt wurden. \cite{kassenarztlichebundesvereinigungRichtlinieKassenarztlichenBundesvereinigung2019} \\

Nach Angabe des Quartals erfolgt eine Überprüfung der BSNRs, die in diesem Quartal für den Vertragsarzt gültig sind. 
Dabei wird geprüft, an welchen Betriebsstätten der Vertragsarzt tätig war. 
Nach Erhalt der BSNRs kann der Vertragsarzt eine BSNR aussuchen, für die diese Erklärung erstellt werden soll. 


\begin{figure}[H]
    \center 
    \includegraphics[width=1\textwidth]{pictures/VJE.PNG} 
    \caption{Vierteljahreserklärung }
    \label{fig:vje}
\end{figure}
Die App \textit{Vierteljahreserklärung} besteht aus einer C\#-Anwendung und einer Angular-Anwendung. 
Dabei ist die Angular-Anwendung für die Darstellung der Daten zuständig. 
Die C\#-Anwendung sorgt dafür, dass alle Abfragen durchgeführt werden und die notwendigen Daten für die Darstellung zur Verfügung stehen. 
Damit beide Anwendungen miteinander kommunizieren können, wird in der C\# Anwendung eine REST-Schnittstelle zur Verfügung gestellt.
Die Schnittstelle bietet zwei GET-Anfragen: \textit{\glqq /api/bsnrs\grqq{}} und \textit{\glqq /api/pdf\grqq{}}.
Beim Aufruf der GET-Anfrage \textit{\glqq /api/bsnrs\grqq{}} werden zusätzlich noch zwei Query-Parameter benötigt. 
Die Parameter sind das Quartal und das zugehörige Jahr, für die die passenden BSNRs geladen werden sollen. \\
Der Aufruf der GET-Anfrage:\textit{\glqq /api/bsnrs\grqq{}} ruft die C\#-Methode \textit{GetBSNRsAsync} auf.
GetBSNRsAsync liefert die gültigen BSNRs für ein bestimmtes Quartal.  
In der Methode wird die \textit{GetBSNR} Methode aus der \textit{ArztregisterService} Klasse aufgerufen.
Die Methode beschafft die notwendigen Daten der Arztregister-Schnittstelle. 
Die Arztregister-Schnittstelle stellt alle notwendigen Daten der Mitglieder zur Verfügung. 
Die Schnittstelle wird von dem CFT Sicherstellung der KVWL gewartet und weiterentwickelt. 
Die aus der Arztregister-Schnittstelle erhaltenen Daten sind sehr stark verschachtelt und enthalten viele Informationen, die für die Vierteljahreserklärung irrelevant sind. 
Daher müssen die Daten in DTOs (Data transfer Objects) geladen werden, um die Menge an HTTP-Anfragen zu reduzieren. 
Mit den DTOs können dann mehr Daten über einen Aufruf gesendet werden. 
Das Reduzieren der Daten ist für die Anwendungen im Mitgliederportal wichtig, denn nicht jedes Mitglied der KVWL besitzt eine ausreichend starke Netzanbindung, um große Datenmengen zu erhalten. 
Daher ist es wichtig, dass alle Daten, die über das Netzwerk gesendet werden, nur relevante Informationen enthalten.


\section{Konfiguration von NLog}

Mithilfe von NLog können drei Richtlinien der Projektarbeit umgesetzt werden. 
Die Richtlinien sind: 
\begin{itemize}
    \item Die Informationen, die ein Log mitliefert
    \item Strukturiertes Logging  
    \item Die Verwendung des Logging Frameworks NLog
\end{itemize}

Damit die Anforderungen aus den Richtlinien bezüglich NLog vollständig erfüllt werden können, muss NLog richtig konfiguriert werden. 
Die Konfiguration von NLog kann entweder durch eine eigene \textit{Nlog.config} Datei realisiert werden oder durch das Einbinden der Konfigurationen in die \textit{Web.config}.
Um den Zugriff auf die Konfigurationen von NLog zu vereinfachen, wurden die Konfigurationen in einer eigenen \textit{Nlog.config} Datei ausgelagert. 
Dies hat den Vorteil, dass die Konfigurationen für NLog schneller erreicht werden und global von mehreren Anwendungen genutzt werden können. \\
Damit die gewünschte Log-Struktur und strukturiertes Logging verwendet werden können, muss ein \textit{target} definiert werden. 
Dieses \textit{target} beschreibt den Ort, an den die Logs geschrieben werden sollen, und welche Informationen mitgeliefert werden. 
Die gewünschte Struktur erfordert die Nutzung des \textit{JsonLayout}.
Ein Layout ist ein vordefiniertes Format, in dem die Logs ausgegeben werden. \cite{julianverdurmenNLogNLog} \\
Das Layout kann dann mit Attributen erweitert werden, um den Inhalt individuell zu gestalten, denn das Layout bezieht sich nur auf die Form der Daten. In diesem Fall ist es im JSON-Format. 
Das definierte Layout ist in Quellcode \ref{code:nlogConfig} zu sehen. 
In dem Layout werden alle Attribute definiert. 
Die Attribute werden dann in allen Logs mitgeliefert. 
Im Layout wird außerdem definiert, dass strukturiertes Logging verwendet werden soll. 
Durch die Zeile 3, in der \textit{includeAllProperties} auf \textit{true} gesetzt wird, wird das Erweitern der Logs durch eigens erstellte Attribute im Quellcode ermöglicht. 
So ist es dann möglich, für spezielle Fälle die Logs zu erweitern. 

\begin{lstlisting}[language=xml, caption=Konfiguration NLog, label=code:nlogConfig]
<layout type="JsonLayout">
  <attribute name="properties" encode="false">
    <layout type="JsonLayout" includeAllProperties="true" maxRecursionLimit="2">
      <attribute name="time" layout="${longdate}"/>
      <attribute name="correlationId" layout="${activityid}"/>
      <attribute name="method" layout="${callsite}"/>
      <attribute name="level" layout="${level}"/>
      <attribute name="message" layout="${message}"/>
    </layout>
  </attribute>
</layout>
\end{lstlisting}

Um die gewünschten Daten im Log zu erhalten, werden vordefinierte Variablen genutzt. 
Jedoch reichen die vordefinierten Variablen nicht aus. 
Damit die \textit{correlationId} genutzt werden kann, muss die \textit{\$\{activityid\}} zu Beginn eines Requests neu gesetzt werden. 
Das geschieht in der \textit{Global.asax}. 
Die \textit{Global.asax} ist eine globale Anwendungsdatei in ASP.NET-Anwendungen. 
In der \textit{Global.asax} werden seitenübergreifende Ereignisbehandlungsroutinen definiert, wie zum Beispiel der Start einer Sitzung.
\cite{schwichtenbergGlobaleEreignisseASP} \\
Der Quellcode dafür ist in Quellcode \ref{code:correlationId} zu finden. 
Die genutzte Methode wird von der Oberklasse der \textit{Global.asax} vererbt und sorgt dafür, dass die Methode beim Starten eines neuen Requests aufgerufen wird.
Die Oberklasse ist \textit{HttpApplication}.

\begin{lstlisting}[caption=Setzen der CorrelationID, label=code:correlationId]
protected void Application_BeginRequest(object sender, EventArgs e)
{ 
    Trace.CorrelationManager.ActivityId = Guid.NewGuid();
}
\end{lstlisting}

\section{Aufbau der Logs}

Das Ziel der Richtlinien ist das Produzieren von Logs, die lesbarer und effizienter zu durchsuchen sind.
Wichtig beim Lesen und Suchen ist dabei der Aufbau der Logs. 
Die Richtlinien definieren, dass ein Log folgende Informationen liefern soll: 

\begin{itemize}
    \item Timestamp
    \item LogLevel
    \item Correlation ID
    \item System-Komponente 
    \item String, um den Fehler oder das Ereignis zu beschreiben
\end{itemize}

Durch die Konfiguration von NLog werden alle Informationen, die benötigt werden, auch mitgeliefert. 
Ein Log, dass jetzt geschrieben wird, sieht wie folgt aus: 

\begin{lstlisting}
{ 
    "properties": { 
        "time": "2021-01-04 11:47:18.7972", 
        "correlationId": "4f80c77d-30d4-453a-933a-392e502a29e6", 
        "method": "KVWL.PortalVEAddIn.BLL.Services.ArztregisterService.GetBSNR", 
        "level": "Info", 
        "message": "LANR: \"1234567\" in dem Jahr: 2020 Mit diesem Quartal: 4", 
        "LANR": "8303571", 
        "year": 2020, 
        "quarter": 4 
    } 
}
\end{lstlisting}

Der dafür genutzte Log-Befehl ist: 
\begin{lstlisting}
log.Info("LANR: {LANR} in dem Jahr: {year} Mit diesem Quartal: {quarter}", lANR, year, quarter);
\end{lstlisting}

Die Nachricht, die beim Log-Befehl eingegeben wird, ist im Log anschließend im Attribut \textit{message} hinterlegt.
Im Log werden auch die drei definierten Attribute \textit{LANR}, \textit{quarter} und \textit{year} als eigene Attribute gespeichert. 
Alle Attribute die von NLog geschrieben werden, sind unter dem Attribut \textit{properties} gespeichert. 
Der Grund dafür ist der Elastic Stack. 
Denn durch das extra Attribut können die Logs besser in Kibana visualisiert werden. 
Filebeat sendet zusätzlich zu den Logs eigene Daten. 
Durch das zusätzliche Attribut \textit{properties} können die anwendungsspezifischen Daten schneller von den Filebeat spezifischen Daten differenziert werden.


\section{Was wird geloggt?}
In diesem Kapitel werden Beispiele der durchgeführten Logging-Anweisungen gezeigt. 
Dabei werden die Logs in drei Kategorien unterteilt. 
Jede, der in diesem Kapitel vorgestellten Logs sind in der Vierteljahreserklärung durchgeführt worden.
Die Logs werden zusätzlich in die Kategorien eingeordnet, die in der Projektarbeit vorgestellt wurden. \cite{bollichErarbeitungLoggingRichtlinienFuer2020}

\subsection{Aufruf von externen Schnittstellen}

Ein Abschnitt, der in der Vierteljahreserklärung geloggt werden muss, ist der Aufruf von externen Schnittstellen. 
In der Vierteljahreserklärung wird eine Schnittstelle mehrmals genutzt, um Daten der Mitglieder zu erhalten. 
Die Schnittstelle ist die Arztregister-Schnittstelle. 
Die Arztregister-Schnittstelle wird vom CFT Sicherstellung \& Versorgungsqualität verwaltet. 
Die Schnittstelle liefert alle relevanten Stammdaten über die Mitglieder der KVWL. \\

Das Loggen dieser Aufrufe ist wichtig, um zu prüfen, ob die Schnittstelle die erwarteten Daten zurückliefert. 
So kann bei fehlerhaftem Aufruf überprüft werden, ob die Daten noch übereinstimmen oder durch Veränderung der Struktur falsche Daten übermittelt werden. 
Die Aufrufe dieser Schnittstelle werden von einem \textit{ArztregisterClient} durchgeführt, der vom CFT Portale entwickelt wurde. 
Durch diesen Client ist die Kommunikation mit dem Arztregister vereinfacht worden. 
Der Client verhindert, dass die Entwickler manuelle REST-Abfragen schreiben müssen. 
Dafür stellt der Client C\#-Methoden zur Verfügung, die den Aufruf der REST-Abfragen ermöglichen. 
Quellcode \ref{code:aufrufArztregister} zeigt ein Log, das nach dem Aufruf einer vom ArztregisterClient zur Verfügung gestellten Methode geschrieben wird. 
Das Log-Level ist in diesem Fall \textit{Debug}.
Das Log-Level \textit{Debug} wurde hier gewählt, weil die Informationen nur für die Entwicklung relevant sind. 
Im produktiven Betrieb ist es nicht relevant, welche Informationen von dem Befehl zurückgeliefert werden. \\
Dieses Log gehört zu der Kategorie \textit{Semantic Description}, die in der Projektarbeit vorgestellt wurde.
Um präziser zu sein, gehört dieses Log zu der Unterkategorie \textit{variable description}. 
Das Log liefert den Inhalt einer Variable, um zu überprüfen, ob die Variable den erwarteten Wert beinhaltet. \cite{bollichErarbeitungLoggingRichtlinienFuer2020}

\begin{lstlisting}[caption=Aufruf der Arztregister-Schnittstelle, label=code:aufrufArztregister]
var leistungserbringerListe = await _arztregisterClient.GetLeistungserbringerClusterAsync(lanr: lANR);
log.Debug("Erhaltene Leistungserbringerliste: leistungserbringerliste={leistungserbringerliste}", leistungserbringerListe);
\end{lstlisting}


\subsection{Auftreten von Fehlern}

Logs werden oft genutzt, um Fehler anzeigen und analysieren zu können. 
Daher wurden in der Vierteljahreserklärung Error-Messages geloggt. 
Dafür wurde das Log-Level \textit{Error} genutzt. 
Quellcode \ref{code:auftretenFehler} zeigt die Methode \textit{ValidatePflichtangaben}. 
Diese Methode startet die Validierung von \textit{Pflichtangaben}.
Wenn die Validierung fehlgeschlagen ist, wird ein error-log geschrieben und eine Exception wird geworfen. \\
Das Loggen dieser Aufrufe ist wichtig, um zu verstehen, welche Fehler während der Laufzeit aufgetreten sind, damit diese anschließend analysiert werden können. 
Fehleranalyse ist ein wichtiger Teil des Log-Managements. \\
Die Art von Logs gehören zu der Kategorie \textit{Error Message}. 
Die genaue Unterkategorie ist in diesem Fall \textit{value-check}. \\

\begin{lstlisting}[caption=Auftreten von Fehlern, label=code:auftretenFehler]
private void ValidatePflichtangaben(Pflichtangaben pflichtangaben)
  {
    var pflichtangabenValidator = new PflichtangabenValidation();
    var validateResult = pflichtangabenValidator.Validate(pflichtangaben);
    if (!validateResult.IsValid) {
      log.Error("Die BSNR {bsnr} ist nicht im richtigen Format. Die BSNR muss 9-stellig sein und darf nur aus Ziffern besten.", pflichtangaben.Bsnr);
      throw new VierteljahreserklaerungException("Die BSNR ist nicht im richtigen Format. Die BSNR muss 9-stellig sein und darf nur aus Ziffern besten.");
    }
  }
\end{lstlisting}



\subsection{Kommunikation zwischen Server und Client}

In Kapitel \ref{kap:vje} wurde die Client-Server-Architektur der Vierteljahreserklärung vorgestellt. 
Die Kommunikation zwischen Server und Client beschränkt sich dabei auf Anfragen, die der Client an den Server sendet.
Das Senden von Anfragen wird durch Operationen ausgelöst, die von einem Mitglied durchgeführt werden. 
Im Falle der Vierteljahreserklärung wären die Operationen, das Auswählen eines Quartals oder das betätigen des Buttons \glqq PDF Erstellen\grqq{}.
Damit nachvollzogen werden kann, welche Operationen ein Mitglied ausgeführt hat, muss die Kommunikation zwischen Server und Client geloggt werden. 
Die Identifizierung eines Mitglieds, anhand der Logdateien, kann den Fehleranalyseprozess vereinfachen. 
Beim Erhalt eines Fehlers durch ein Mitglied kann das entsprechende Log problemlos identifiziert werden. 
Damit diese Identifizierung möglich ist, müssen alle Anfragen, die der Client an den Server sendet, geloggt werden. \\
Da der Fehleranalyseprozess im produktiven Betrieb durchgeführt wird, müssen die Logs im Log-Level \textit{Info} geschrieben werden. 
Quellcode \ref{code:aufrufREST} zeigt die Methode \textit{GetBSNRsAsync}. 
Diese Methode wird aufgerufen, wenn ein Quartal in der Vierteljahreserklärung ausgewählt wurde. 
Der Client sendet die Anfrage, um die gültigen BSNRs des Mitglieds für den ausgewählten Zeitraum zu erhalten.
Das dazu geschriebene Log befindet sich zu Beginn dieser Operation, um zu definieren, dass der Aufruf der Methode geklappt hat. \\
Da das Log zu Beginn der Operation platziert wurde, gehört dieses Log zu der Kategorie \textit{Program Operation} und zusätzlich zu der Unterkategorie \textit{next operation}.


\begin{lstlisting}[caption=Aufruf der REST-Schnittstellen, label=code:aufrufREST]
[Route("api/bsnrs")]
public async Task<IHttpActionResult> GetBSNRsAsync(int quarter, int year)
{
  log.Info("Aufruf der REST-Schnittstelle: GetBSNRsAsync mit: quarter={quarter}, year={year}", quarter, year);
  var bsnrList = await _arztregisterService.GetBSNR(quarter, year, LANR);
  return Ok(bsnrList);
}
\end{lstlisting}

