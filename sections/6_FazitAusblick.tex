\chapter{Fazit und Ausblick}
\label{fazit}

Das Ziel dieser Projektarbeit war es, Logging-Richtlinien für das CFT Portale zu definieren.
Mithilfe dieser Richtlinien soll der Aufwand des CFT Portale minimiert werden.
Um dies zu schaffen, wurden zu Beginn der Projektarbeit die wichtigsten Bestandteile des Loggings näher erläutert. 
Dabei wurden schon Aspekte erwähnt, die im späteren Verlauf immer wichtiger wurden. 
Einen zentralen Aspekt stellt das zentralisierte Logging dar. 
Damit wird gewährleistet, dass eine der aufwändigsten Punkte des Bugtrackings behoben wird: Die Zusammenführung örtlich getrennter Logfiles. \\
Anschließend mussten die Technologien des CFT Portale analysiert werden. 
Dabei wurden Entwicklungstools und Frameworks, sowie Software-Architekturen genauer untersucht. 
Der nächste Schritt war die Definition von den Richtlinien. 
Hier wurden sechs Richtlinien definiert, an denen sich das Team in Zukunft orientieren soll. 
Die Richtlinien beziehen sich hierbei auf die geloggten Informationen, die Art des Loggings, das Logging-Framework, die genutzten Log-Level, die Art der Speicherung und die dazugehörigen Tools, sowie die Dauer der Speicherung.
Als Abschluss dieser Projektarbeit wurden Logging-Herausforderungen beschrieben. 
Dabei wurde Anti-Pattern und \textit{Logging code smells} vorgestellt und näher erläutert. \\


Im Laufe dieser Projektarbeit ist aufgefallen, dass das Logging-Konzept des CFT Portale chaotisch und unstrukturiert ist. 
Das Team nutzt ein veraltetes Framework das nicht mehr aktiv gewartet wird und es wird auch kein zentrales Logging verwendet. 
Generell kann gesagt werden, dass keine Struktur im Logging des Teams zu erkennen ist. 
Durch die in dieser Projektarbeit erstellten Richtlinien kann das Team strukturiert arbeiten.  \\

Im Anschluss an diese Projektarbeit folgt eine Bachelorarbeit. 
Diese wird die hier theoretisch erfassten Inhalte in die Praxis umsetzen. 
Es sollen unterschiedliche Tools zur zentralisierung der Logs Evaluiert werden. 
So kann gezeigt werden, dass der Elastic Stack die beste Lösung für das Team ist oder eine bessere Lösung wird gefunden. 
Nach der Evaluation soll ein zentralisiertes Logging entwickelt werden. 
Entweder mit dem Elastic Stack oder mit einem anderen.
So können dann die Logs des Teams zentral gespeichert werden. 
Anschließend sollen in einer Anwendung des CFT Portale die erstellten Logging-Richtlinien umgesetzt werden. 



