\chapter{Aktueller Stand im CFT Portale}
\label{situationCFT}

Zum Aufgabenbereich des CFT Portale gehören die Entwicklung neuer Anwendungen, als auch die Weiterentwicklung bestehender Anwendungen.
Nicht nur die Entwicklung, sondern auch die Administration der Software wird vom Team selber durchgeführt. 
Daher besitzt das Team eine Reihe unterschiedlichster Tools, um die vielfältigen Aufgaben lösen zu können. \\

Das Team wird bei der Entwicklung regelmäßig von externen Mitarbeitern begleitet. 
Dies sorgt für eine hohe Anzahl an Entwicklern, die an einer Software arbeiten. 
Die daraus entstandene Flut an unterschiedlichsten Programmierstilen, hat das Software-Angebot unübersichtlich und unstrukturiert werden lassen. \\

Das Team hat sich in den letzten zwei Jahren stark gewandelt. 
Durch den Einfluss und das Engagement von neuen und bestehenden Mitarbeitern hat sich der Entwicklungsprozess verbessert.
Die Wichtigkeit von Code-Qualität und standardisierten Abläufen ist stark gestiegen. \\

Um den aktuellen Stand vorzustellen, werden alle wichtigen Aspekte in der Entwicklung vorgestellt und die einzelnen Probleme in Bezug auf das Logging erläutert. 
Dabei wird ein Blick auf den Technologie-Stack geworfen, sowie auf den Entwicklungsprozess und die vorhandenen Software-Architekturen.

\section{Technologie-Stack}

Das CFT Portale nutzt einige Tools, die für diesen Zweck näher erläutert werden müssen. 
So kann besser verstanden werden, welche Möglichkeiten bestehen, um ein effektives Logging zu erreichen und welche Hürden unterschiedliche Technologien mit sich bringen können. 

    \subsection{Angular}

    Angular ist ein Open-Source-Framework, das von Google entwickelt wurde, um die Entwicklung von Single-Page Web-Applikationen zu ermöglichen.
    Dabei nutzt das Framework die Programmiersprache TypeScript. 
    TypeScript übernimmt bei Angular die Rolle, die in herkömmlichen Web-Anwendungen durch JavaScript besetzt wird. 
    Die Darstellung der einzelnen Komponenten wird mit HTML und CSS durchgeführt. \cite{googleAngularIntroductionAngular} 

    \subsection{Microsoft .NET Framework}

    Das CFT Portale nutzt in ihren Applikationen das .NET Framework.
    Das .NET Framework besteht aus Entwicklungswerkzeugen, Schnittstellen und auch Klassenbibliotheken. 
    Im gegensatz zum .NET Framework ist das .NET Core nicht nur für Windows sondern auch für Linux und macOS verfügbar.
    Das .NET Core wird im Team noch nicht genutzt. 
    Mit der Version 5 von .NET wird sich das jedoch ändern, weil es dann möglich sein wird, mit .NET Core Provider-hosted Add-ins zu erstellen. \cite{groteMicrosoftNETFramework}
    Das Team verwendet derzeit die Version 4.8.
    Die genutzte Programmiersprache ist hier C\#.

    \subsection{SharePoint}
    SharePoint ist eine Plattform für Unternehmen, um Websiten zu erstellen, Daten zu speichern, Daten zu strukturieren und auf Daten zugreifen zu können.
    Der Zugriff kann über alle Geräte erfolgen, die über einen Webbrowser verfügen. \\
    In SharePoint können SharePoint-Teamwebsiten erstellt werden. 
    Diese Teamwebsiten können genutzt werden, um Dokumente und andere Daten zu Speichern. 
    Außerdem kann das Team gemeinsam an einem Dokument arbeiten.
    So wird das Kollaborative Arbeiten im Team ermöglicht. \cite{microsoftWasIstSharePoint} \\

    Das CFT Portale stellt den Mitgliedern und den Mitarbeitern der KVWL eine Plattform zur Verfügung, die unterschiedliche Anwendungen beinhaltet. 
    Dafür nutzt das Team den SharePoint. 
    Der SharePoint ist dabei der Einstiegspunkt für die Mitarbeiter und die Mitglieder.  
    Das Team bietet zwei Plattformen, einmal das Intranet für die Mitarbeiter und noch das Mitgliederportal für die Mitglieder. 
    Im Intranet der KVWL werden Dateien, News und auch ein Projekt-Center angeboten. 
    Das Projekt-Center ermöglicht das gemeinsame Arbeiten an Dokumenten und eine Art Wiki für die Projekte.
    Im Mitgliederportal werden nur die provider-hosted Add ins zur Verfügung gestellt.
    Die Kollaborations-Funktion des SharePoint kommt hier nicht zum Einsatz.
    

    \subsection{Dynatrace application performance management}

    Dynatrace application performance management (APM) ist für die Überwachung der IT-Infrastruktur verantwortlich. 
    Dabei misst sie mithilfe einer KI den Ablauf von Hardware und Software und kann somit Fehler erkennen und anzeigen. 
    Alle gesammelten Informationen werden in Diagrammen dargestellt. \cite{dynatraceApplicationPerformanceManagement} \\

    Das CFT Portale, nutzt Dynatrace um ihre Software Monitoren zu können.  
    Mithilfe der Diagramme kann das CFT schnellstmöglich erkennen, wenn ein System ausgefallen ist oder irgendwelche Zugriffe nicht mehr möglich sind. 
    

\section{Software-Architektur}
% Hier erlkären, kein eigenes Kapitel \subsubsection{Dev-Ops}

Die vom Team erstellte Software, basiert auf der Client-Server Architektur. 
Das heißt, die Aufgaben, die von der Software bearbeitet werden müssen, werden auf Clients und Server verteilt. 
Der Client ist dabei für die Darstellung, die Validierung und den Input der Daten  zuständig. 
Der Server kümmert sich um das Versorgen der Clients mit den benötigten Daten, die Speicherung von neuen Daten in der Datenbank und enthält Business-Logik. 
Außerdem kümmert sich der Server um alle Berechnungen die durchgeführt werden müssen. 

    \subsection{Client-Server Architektur}

    Die Client-Server Architektur beschreibt ein Modell der Kommunikation in Software-Anwendungen. 
    Dabei stellt der Client Anfragen an den Server, die Anfragen werden dann vom Server ausgewertet und liefern die gewünschten Daten. 
    Ein Server ist für mehrere Clients verantwortlich. \\
    Im Falle des CFT Portale, spiegelt der Client das Frontend der Anwendung wider. 
    Die Rolle des Servers übernimmt dabei das Backend. \cite{it-administrator.deClientServerArchitekturItadministrator}

    \subsection{Frontend/Client} 
    Der Client wird im CFT Portale mit dem Angular Framework erstellt. 
    Bei der Vielzahl an Anwendungen bestehen jedoch unterschiedliche Versionen von Angular. 
    Ältere Software, die noch nicht aktualisiert wurde, besteht noch aus dem Angular JS Framework. 
    Das Angular JS Framework wurde von dem aktuellen Angular 2+ ersetzt. 
    Dabei handelt es sich um die Änderung, dass der JavaScript (JS) -Teil von TypeScript (TS) abgelöst wurde. \cite{googleAngularIntroductionAngular} \\

    Das Frontend ist für die Darstellung und das Senden von Abfragen an den Server verantwortlich. 
    Um zwischen dem Client und dem Server eine Kommunikation zu ermöglichen, werden REST-Schnittstellen bereitgestellt. \\
    Derzeit werden die vom Client erzeugten Logs nicht an den Server gesendet. 
    Das bedeutet, dass alle im Client erzeugten Logs nur in der Konsole des Browsers zu sehen sind. 
    Eine Log-Analyse für den Client ist nicht möglich. 

    \subsection{Backend/Server}

    Das Backend wird in der Programmiersprache C\# entwickelt. 
    Dies ermöglicht die Nutzung von SharePoint-Komponenten.
    Das Team nutzt, wie im letzten Kapitel beschrieben, nicht viele Komponenten des SharePoint.
    Im Mitgliederportal werden nur die Listen vom SharePoint genutzt.
    Im Intranet hingegen werden mehrere Komponenten verwendet. 
    Dazu gehören zum Beispiel Kalender, Bilderbibliotheken und Dokumentenbibliotheken. \\
    
    Im Backend werden alle logischen und berechnenden Aufgaben bearbeitet.
    Dazu gehörten zum Beispiel das Verwalten von benutzerspezifischen Daten. 
    Das Backend muss über mehrere Anfragen an andere Schnittstellen Daten sammeln, die dann in ein Objekt für das Frontend passend zusammen gestellt werden. 
    Dies muss geschehen, weil die vom Backend erhaltenen Daten, sehr viel mehr Informationen beinhalten, als notwendig. 
    Der Vorteil dieser Art von Transfer besteht in der Einsparung von Netzwerkkapazitäten. 
    Nicht alle Mitglieder der KVWL besitzen eine starke Netzwerkbandbreite, daher wird versucht den Datentransfer niedrig zu halten. \\

    Das Logging im Backend erfolgt derzeit ohne Struktur. 
    Einige Entwickler haben angefangen, das Logging-Framework Log4Net zu nutzen. 
    Der Einsatz erfolgt jedoch ohne Absprache und nur nach eigenen Bedarf der Entwickler. 
    Also können die erstellten Logfiles nur für die Entwicklung genutzt werden. 
    Der Nutzen in der Produktion ist noch nicht auf gewünschten Stand.
    
    \subsection{Provider-hosted Add-in}

    Die erstellte Web-Anwendung wird als \textit{Provider-hosted Add-in} betrieben. 
    Ein Provider-hosted Add-in ist eine Anwendung die zwar auf einer SharePoint Farm registriert ist, jedoch nicht an die von Microsoft vorhanden Werkzeuge gebunden ist. 
    Die erstellten Anwendungen können auf einer eigenen Umgebung komplett losgelöst vom SharePoint genutzt werden.  
    Der SharePoint ist für die Darstellung der entwickelten Anwendungen da. 
    Sie haben alle die Masterpage des SharePoint. 
    So haben alle erstellten Anwendungen die gleiche GUI. \cite{microsoftErsteSchritteBeim} \\

    SharePoint besitzt einen eigenen Logging-Mechanismus.
    Durch die Nutzung von Provider-hosted Add-ins ist werden die Logs nicht durch den SharePoint erstellt und konfiguriert.
    Daher müssen die einzelnen Provider-hosted Add-ins selber Logs erstellen. 

    \subsection{Redundante Serververteilung}
    \label{redundanteSerververteilung}

    Damit der Betrieb bei hohen Lasten weiterhin gewährleistet werden kann, werden im CFT Portale alle Anwendungen redundant auf zwei Servern deployed. 
    So können Nutzer bei hoher Last auf den zweiten Server geleitet werden.
    Dies ermöglicht, dass bei den Abrechnungsperioden alle Anwendungen weiterhin erreichbar bleiben. 
    Außerdem sind die Anwendungen durch die redundante Verteilung Ausfallsicher. \\

    Die Verteilung der Anwendungen bringt für das Logging ein Problem mit sich. 
    Denn Nutzer können dadurch auf den beiden Servern innerhalb eines Prozesses verteilt unterwegs sein.
    So entstehen zwei Logfiles auf unterschiedlichen Servern. 
    Dadurch steigt die Komplexität bei der Fehler Identifikation. 
    Das Reproduzieren von Beginn des Prozesses an bis zum auftreteten des Fehlers ist deshalb schwierig.

\section{Entwicklungsprozess}

Das CFT Portale nutzt das Vorgehensmodell Scrum. 
Das Team hat ein Product Backlog indem alle Anforderungen festgehalten werden. 
Dieses Backlog wird regelmäßig mit neuen Anforderungen (Tickets) befüllt. 
Die Tickets werden in regelmäßigen Sprints abgearbeitet. 
Jedoch werden dabei nicht alle Tickets aus dem Product Backlog abgearbeitet, sondern nur die Tickets aus dem Sprint Backlog. 
Das Sprint Backlog wird im Sprint Planning Meeting erstellt. 
Dabei werden die Tickets aus dem Product Backlog in das Sprint Backlog geschoben.
Welche Tickets im nächsten Sprint bearbeitet werden sollen, entscheidet der Product Owner.
Das Ziel in jedem Sprint ist es, alle Tickets aus dem Sprint Backlog zu erledigen.
Wie viele Tickets das Entwickler Team schaffen kann, entscheidet das Team selber. 
Ein Sprint dauert im CFT Portale zwei Wochen. 
Nach dem Sprint folgt ein Sprint Review und eine Sprint Retrospektive. \\ 

Das Scrum Board besteht aus vier Spalten: \textit{Aufgaben}, \text{Wird Ausgeführt}, \textit{Test} und \textit{Fertig}.\\
Der Prozess im Team sieht vor, dass ein Entwickler sich ein Ticket aus der Spalte \textit{Aufgaben} zieht und dann das Ticket in \textit{Wird Ausgeführt} schiebt. 
Dann wird ein \textit{feature} oder \textit{bug} Branch eröffnet. 
In dem Branch wird dann das Ticket erledigt. \\
Das Team nutzt SonarQube, der den Quellcode überprüft.
Wenn der neu erstellte Code keine Testabdeckung von mindestens 60\% hat, neue Bugs enthält oder neue code smells eingefügt hat, dann wird der geschriebene Code nicht akzeptiert. 
Nachdem der geschrieben Code den Vorgaben entspricht, stellt der Entwickler einen Pull-Request und schiebt das Ticket in die \textit{Test} Spalte. 
Nach erstellen des Pull-Requests muss ein weiterer Entwickler den Code überprüfen und lokal ausführen. 
Wenn alle Tests erfolgreich waren, wird der geschriebene Code mit dem Dev Branch gemerged und das Ticket landet in der \textit{Fertig} Spalte. 


\section{Software-Angebot}
Im CFT Portale werden die unterschiedlichsten Anwendungen betreut. 
Dabei gibt es zwei Produkte, unter denen viele weitere Anwendungen stehen. 
Die Produkte sind: das \textit{Mitgliederportal} und das \textit{Intranet}.
Das Mitgliederportal ist für die Mitglieder der KVWL und das Intranet für die Mitarbeiter der KVWL.
Die Mitglieder der KVWL sind niedergelassene Ärzte und Psychotherapeuten in Nordrhein-Westfalen.

    \subsection{Mitgliederportal}

    Im Mitgliederportal der KVWL werden Services für die Mitglieder der KVWL angeboten. 
    Im Portal können die Mitglieder ihre persönlichen Daten prüfen und gegebenenfalls ändern. 
    Außerdem können dort z.B. Quartalsabrechnungen erstellt und abgeschickt werden, sowie Anträge, die den Mitgliedern zum Beispiel Geld zurückerstatten.
    Die Quartalsabrechnungen sind die Leistungen, die die Mitglieder in einem Quartal erbracht haben. \\

    Das Mitgliederportal besteht ausschließlich aus Provider-hosted Add-ins. 
    Es werden acht Anwendungen im Mitgliederportal zur Verfügung gestellt.
    Jede dieser Anwendungen läuft unabhängig vom SharePoint auf eigenen Servern. 
    In den Listen vom SharePoint sind die Daten der einzelnen Anwendungen hinterlegt.
    So können auf der Startseite vom Mitgliederportal Kacheln angezeigt werden, die zu den einzelnen Anwendungen weiterleiten.

    \subsection{Intranet} 
    Das Intranet der KVWL bietet den Mitarbeitern unterschiedliche Funktionen an. 
    Darunter befinden sich eine News-Plattform, Speisepläne, Stellenausschreibungen, ein Telefonbuch und auch eine Sammlung von wichtigen Dokumenten. 
    Außerdem findet man auch Links zu Plattformen, wie die der Zeiterfassung und der Raumplanung. 
    Generell ist das Intranet eine Ansammlung von Informationen und Anwendungen, die jeder Mitarbeiter im Laufe des Tages gebrauchen kann. \\

    Die Entwicklung des Intranets besteht nicht wie im Mitgliederportal nur aus Provider-hosted Add-ins, sondern auch aus sogenannten SharePoint Driven Apps.
    Dabei werden nur Frontend Applikationen entwickelt, die in kompilierter Form in den SharePoint eingebunden werden. 
    Diese Apps sind leichtgewichtiger und ermöglichen eine schnelle und unkomplizierte Entwicklung. \cite{packof7SharePointDrivenApplication}



\section{Überblick}

Dieses Kapitel beschreibt die aktuelle Situation des CFT Portale. 
Zu Beginn wurden die genutzten Technologien vorgestellt. 
Dazu gehören Angular, \textit{.NET}, SharePoint und Dynatrace.
Im Anschluss darauf wurde die Software-Architektur genauer untersucht. 
Dabei ist aufgefallen, dass eine Anwendung aus einem Front- und Backend besteht. 
Die Anwendungen werden dann als Provider-hosted Add-ins deployed und im SharePoint zur Verfügung gestellt. 
Zum Schluss wurden noch die beiden wichtigen Anwendungen des Teams vorgestellt.
