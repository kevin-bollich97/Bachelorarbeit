\chapter{Ergebnisse}

In diesem Kapitel wird das Ergebnis der durchgeführten Installationen und Anpassungen des Quellcodes vorgestellt. 
Dabei wird zu Beginn das Problem des vorherigen Prozesses näher erläutert, um genauer festzustellen, was die Veränderungen hinsichtlich Fehlersuche gebracht haben. 
Dies wird anhand eines Beispiels verdeutlicht.


\section{Fehleranalyse im Ausgangszustand}

Das CFT Portale muss bei der Fehleranalyse auf einen langwierigen Prozess setzen.
Dieser Prozess ist in Abbildung \ref{fig:prozessVorher} anhand eines Aktivitätsdiagramms dargestellt. 
Der Prozess beginnt mit dem Erhalt eines Tickets oder einer E-Mail, der einen Fehler in der Vierteljahreserklärung beschreibt. 
Nachdem der Entwickler sich das Ticket näher angeschaut hat, muss dieser nach dem dabei entstandenen Log suchen. 
Dafür muss sich der Entwickler an einem der beiden PHA-Server anmelden und die Logs manuell durchsuchen. 
Dabei wird eine Remotedesktopverbindung auf dem ersten PHA1-Server aufgebaut.
Anschließend muss die entsprechende Log-Datei für die Vierteljahreserklärung in der Ordnerstruktur des Servers gefunden werden. 
Nachdem die Log-Datei geöffnet wurde, muss die textbasierte-Datei manuell nach dem entsprechenden Log durchsucht werden. 
Bei der Suche kann der Fall auftreten, dass der entsprechende Fehler nicht auf diesem PHA-Server entstanden ist. 
Das bedeutet, dass der Fehler noch auf dem zweiten PHA-Server gesucht werden muss. 
Daher müssen die Schritte \textit{Suchen der Datei} und \textit{Durchsuchen der Logs} wiederholt werden. 
Wenn der Fehler auf dem zweiten PHA-Server entdeckt wurde, kann er analysiert und anschließend dokumentiert werden. 
Der Prozess ist mit der Dokumentation abgeschlossen. 


% Diagramm zeichnen
% Aktivitätsdiagramm?
% Logs zeigen

\begin{figure}[H]
    \center 
    \includegraphics[width=1\textwidth]{pictures/ProzessVorher.png} 
    \caption{Ausgangszustand der Fehleranalyse}
    \label{fig:prozessVorher}
\end{figure}

\section{Verbesserter Fehleranalyseprozess}

Der verbesserte Fehleranalyseprozess, der durch das zentralisierte Logging und die Veränderungen des Logging-Codes ermöglicht wurde, fällt im Vergleich zum vorherigen Prozess deutlich einfacher aus. 
Denn schon anhand der beiden Abbildungen \ref{fig:prozessVorher} und \ref{fig:prozessDanach} ist erkennbar, dass der verbesserte Fehleranalyseprozess mit weniger Aktionen auskommt. 
Die Aktionen im vorherigen Prozess benötigen mehr Aufwand für die Durchführung als der verbesserte. \\

Beide Prozesse starten mit dem Erhalt eines Tickets. 
Da fallen die Unterschiede der beiden Prozesse auf: Beim vorherigen Prozess musste sich mit einer Remotedesktopverbindung auf den entsprechenden Server zugeschaltet werden, beim verbesserten Prozess muss nur noch eine URL mit dem Browser aufgerufen werden. 
So wird dann Kibana aufgerufen. 
Der einzige Schritt, der jetzt noch durchgeführt werden muss, ist die Filterung nach den Kriterien. 
Der entsprechende Log-Eintrag taucht dann in der Suche auf. 
Der Fehler kann nun analysiert und dokumentiert werden.  \\

Der größte Unterschied der beiden Abbildungen ist das Fehlen der Schleife in der Abbildung \ref{fig:prozessDanach}.
Das heißt, der verbesserte Prozess muss keine Aktionen mehrmals ausführen, um ans Ziel zu gelangen. 
Im vorherigen Prozess kann dies jedoch geschehen, sollten die gesuchten Logs nicht im \textit{PHA1} vorhanden sein. 
Wenn dies der Fall ist, dann muss die Anmeldung am Server, das Öffnen der Log-Datei und das Durchsuchen der Logs zweimal durchgeführt werden. 
Das erhöht den Aufwand beim Suchen nach bestimmten Log-Einträgen. \\
Nicht nur das mehrfache Ausführen von Aktionen sind Unterschiede der beiden Prozesse, sondern der Aufwand je Aktion ist ebenfalls unterschiedlich. 
Denn schon das Durchsuchen der Logs ist bei dem vorherigen Prozess deutlich aufwändiger.
Dies resultiert daraus, dass die Log-Datei aus dem vorherigen Prozess nicht nach bestimmten Attributen gefiltert werden kann und auch kein Ausblenden von irrelevanten Informationen möglich ist. 
Im Vergleich dazu ist es in Kibana möglich, nach bestimmten Kriterien zu suchen und nur die Informationen anzuzeigen, die relevant sind. 


\begin{figure}[H]
    \center 
    \includegraphics[width=1\textwidth]{pictures/ProzessDanach.png} 
    \caption{Verbesserte Fehleranalyseprozess }
    \label{fig:prozessDanach}
\end{figure}


\section{Erste praktische Erfahrungen}

Die Ergebnisse der Bachelorarbeit wurden dem CFT Portale vorgestellt und zusammen mit den Entwicklern ebenfalls getestet. 
Dabei fiel auf, dass die Oberfläche von Kibana nicht intuitiv genutzt werden kann ohne eine Einweisung. 
Das liegt an der Benennung der Einstiegspunkte für die Logs.
Denn um auf die Logs zugreifen zu können, muss der Unterpunkt \glqq Discover\grqq{} aufgerufen werden. 
Dies ist nicht ohne Ausprobieren zu erahnen. 
Jedoch fiel der erste Eindruck sehr positiv auf, denn nach einer kleinen Einführung in die neuen Produkte wurde deutlich, dass neue Möglichkeiten dadurch entstanden sind. 
Der neue Fehleranalyseprozess fällt deutlich kleiner aus als der ehemalige. \\
Ein wichtiger Punkt, der bei den Entwicklern Eindruck hinterlassen hat, ist die neue Konfiguration mittels NLog im Quellcode der Vierteljahreserklärung. 
Die neue Konfiguration kann ohne Weiteres in andere Anwendungen integriert werden. 
Dafür muss die neue Konfigurationsdatei von NLog in die Anwendungen integriert werden und das passende NuGet Package für NLog muss installiert werden. 


