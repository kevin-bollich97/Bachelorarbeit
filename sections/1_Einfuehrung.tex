\chapter{Einführung}
\label{einfuehrung}

\section{Motivation}
Das cross-funktionale Team Portale (CFT Portale) der Kassenärztlichen Vereinigung 
Westfalen-Lippe verwaltet eine hohe Anzahl an Applikationen, 
bei denen regelmäßig neue Funktionen hinzukommen.
Während der Laufzeit können Fehler auftreten, deren Herkunft nicht immer eindeutig ist. 
Damit effizient nachvollzogen werden kann, wie der Fehler entstanden ist, sollten bestimmte Informationen 
während der Laufzeit geloggt werden. \\ 
Die derzeitige Logging-Situation im CFT Portale ist chaotisch und unstrukturiert.
Im Team existieren keine Richtlinien zum Logging.
Dies bewirkt, dass jeder Entwickler seine eigenen Vorstellungen zum Logging umsetzt. \\
Durch den Mangel an Struktur koexistieren unterschiedliche Logging-Tools, die in der Software des Teams vorhanden sind. 
Jedoch ist nicht nur die Tool-Auswahl ein Problem für die Entwickler, sondern auch die in den Logdateien enthaltenen Informationen. 
Jeder Entwickler entscheidet selbst welche Informationen relevant sind. 
Mit klaren Richtlinien kann dem Team Struktur gegeben werden. 
Durch diese Struktur, können die Entwickler fehlerhaftes Verhalten einer Software effizienter Suchen und Eingrenzen.

\section{Ziel}
Ziel dieser Projektarbeit ist es, Logging-Richtlinien für das CFT Portale zu entwickeln, die die Arbeit der Entwickler erleichtern. 
Um für das CFT Portale passende Logging-Richtlinien zu erstellen, sollen die Applikationen zuerst analysiert werden. \\
Mithilfe der Logging-Richtlinien können sich die Entwickler an eine Vorgabe halten und dabei die Log-Ausgaben der Anwendungen so gestalten, dass ein einheitlicher Aufbau entsteht und alle für das Bugtracking relevanten Informationen vorliegen. 
So soll der Wartungsaufwand im CFT Portale gesenkt werden. \\
Ein weiteres Ziel dieser Projektarbeit soll es sein, dass dem Team negativ Beispiele genannt werden. 
Mit negativ Beispielen sind in dieser Arbeit Anti-Pattern und Logging code smells gemeint. 
Diese sollen zum Ende der Projektarbeit vorgestellt werden. 
Durch die Beispiele ist es den Entwicklern des CFT Portale möglich, die gezeigten Fehler zu vermeiden. \\

%\section{Problemstellung}
%Ab hier alles Problemstellung
Das CFT Portale erweitert sein Softwareangebot stetig weiter. 
Durch die stetige Weiterentwicklung treten Bugs auf, deren Herkunft unbekannt sein kann.
Das Bugfixing ohne eine verständliche und einheitliche Logging-Architektur ist für die Entwickler eine Herausforderung. \\
Die Entwickler müssen beim Auftreten eines Bugs, die Anwendung debuggen und eine aufwendige Fehler-Ursachen-Analyse durchführen. 
Dieser Ansatz hilft nicht immer, denn es können Bugs auftreten, die nicht im direkten Bezug zur Software stehen. 
Fehler können zum Beispiel durch fehlerhafte Hardware oder auf Grund von Netzwerkproblemen auftreten.
Durch die in den Logdateien gespeicherten Informationen, können Entwickler nach Auftreten eines Bugs,
die nötigen Informationen erhalten, um eine effiziente Fehleranalyse durchzuführen. \\


Daraus leiten sich folgende Forschungsfragen ab:
\begin{itemize}
	\item Welche Informationen sollten geloggt werden? 
	\item Wo sollten Logfiles gespeichert werden? 
	\item Wie lange sollten Logdateien gespeichert werden? 
	\item Welche Logging-Frameworks können die Anforderungen des CFT Portale erfüllen? 
	\item Wie lässt sich die Effizienz beim Bugtracking verbessern?
	\item Wie sollte nicht geloggt werden? 
\end{itemize}

\section{Vorgehensweise}

Zu Beginn der Projektarbeit werden allgemeine Informationen zum Logging erläutert, um eine Grundlage für die Inhalte in dieser Arbeit zu schaffen. \\
Anschließend soll die derzeitige Situation im CFT Portale genauer beschrieben werden, damit die relevanten derzeitigen Probleme erkannt werden können.\\
Nachdem die Probleme identifiziert wurden, kann nach Lösungen gesucht werden, die dem CFT Portale zukünftig als Richtlinien dienen sollen. \\
Damit die Entwickler beim Logging keinen schlecht wartbaren Code produzieren, werden noch schlechte Verhaltensweisen beim Logging gezeigt. 
Dazu gehören Anti-Pattern und Logging code smells. 
Zum Schluss wird ein Fazit zum Verlauf der Projektarbeit gezogen, in dem die erlangten Kenntnisse und 
die gefundenen Lösungen noch einmal vorstellt und bewertet werden. 
Außerdem soll ein kurzer Ausblick gegeben werden. \\

%\subsection{CFT Portale}